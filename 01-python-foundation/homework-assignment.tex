\documentclass{article}
\usepackage[utf8]{inputenc}
\usepackage{listings}
\usepackage{xcolor}
\usepackage{hyperref}
\usepackage{geometry}
\usepackage{enumitem}
\geometry{margin=1in}

% Python code style
\definecolor{codegray}{rgb}{0.95,0.95,0.95}
\definecolor{keyword}{rgb}{0.0,0.0,0.6}
\definecolor{comment}{rgb}{0.0,0.5,0.0}
\definecolor{string}{rgb}{0.64,0.08,0.08}

\lstdefinestyle{codeStyle}{
    language=Python,
    backgroundcolor=\color{codegray},
    commentstyle=\color{comment}\itshape,
    keywordstyle=\color{keyword}\bfseries,
    stringstyle=\color{string},
    basicstyle=\ttfamily\small,
    breaklines=true,
    frame=single,
    showstringspaces=false,
    tabsize=4,
    morekeywords={self, True, False, None, with, as, match, case}
}

\title{Python for Cheminformatics \& Bioinformatics: Homework Assignment}
\author{AI-Driven Drug Development Training}
\date{February 2026}

\begin{document}

\maketitle

\section*{Instructions}
\begin{itemize}
    \item Complete all problems in a single Python file named \texttt{homework\_solutions.py}
    \item Use comments to separate each problem (e.g., \texttt{\# Problem 1})
    \item Include docstrings for all functions
    \item Test your code before submission
    \item Total Points: 100
    \item Estimated Time: 3-4 hours
\end{itemize}

\vspace{1em}
\hrule
\vspace{1em}

% ============================================
% PART 1: BIOINFORMATICS BASICS (25 points)
% ============================================
\section{Part 1: Bioinformatics - DNA/RNA/Protein (25 points)}
\textit{Inspired by Rosalind.info problems}

\subsection*{Problem 1: Counting DNA Nucleotides (5 points)}
\textit{(Rosalind ID: DNA)}

Given a DNA string, count the occurrences of each nucleotide.
\begin{lstlisting}[style=codeStyle]
dna = "AGCTTTTCATTCTGACTGCAACGGGCAATATGTCTCTGTGTGGATTAAAAAAAGAGTGTCTGATAGCAGC"
\end{lstlisting}

Implement \texttt{count\_nucleotides(dna)} that returns a dictionary:
\begin{enumerate}[label=\alph*)]
    \item Count of A, C, G, T
    \item Validate input contains only valid nucleotides
    \item Handle both uppercase and lowercase input
    \item Return counts in order: A, C, G, T
\end{enumerate}

\textbf{Expected output:} \texttt{\{"A": 20, "C": 12, "G": 17, "T": 21\}}

\subsection*{Problem 2: Transcribing DNA to RNA (5 points)}
\textit{(Rosalind ID: RNA)}

Create a function \texttt{transcribe(dna)} that converts DNA to RNA:
\begin{enumerate}[label=\alph*)]
    \item Replace all occurrences of 'T' with 'U'
    \item Validate input is valid DNA
    \item Handle edge cases (empty string, invalid characters)
\end{enumerate}

\begin{lstlisting}[style=codeStyle]
dna = "GATGGAACTTGACTACGTAAATT"
# Expected: "GAUGGAACUUGACUACGUAAAUU"
\end{lstlisting}

\subsection*{Problem 3: Reverse Complement (5 points)}
\textit{(Rosalind ID: REVC)}

Create \texttt{reverse\_complement(dna)} that returns the reverse complement:
\begin{enumerate}[label=\alph*)]
    \item A $\leftrightarrow$ T, C $\leftrightarrow$ G
    \item Reverse the complemented string
    \item Handle invalid input gracefully
\end{enumerate}

\begin{lstlisting}[style=codeStyle]
dna = "AAAACCCGGT"
# Expected: "ACCGGGTTTT"
\end{lstlisting}

\subsection*{Problem 4: Computing GC Content (5 points)}
\textit{(Rosalind ID: GC)}

Create \texttt{gc\_content(dna)} that calculates the GC percentage:
\begin{enumerate}[label=\alph*)]
    \item GC content = (G + C) / total length $\times$ 100
    \item Return as percentage with 2 decimal places
    \item Also implement \texttt{highest\_gc(fasta\_dict)} that takes multiple sequences and returns the one with highest GC content
\end{enumerate}

\begin{lstlisting}[style=codeStyle]
sequences = {
    "Rosalind_0001": "AGCTATAG",
    "Rosalind_0002": "GCGCGCGC",
    "Rosalind_0003": "ATATATATAT"
}
\end{lstlisting}

\subsection*{Problem 5: Translating RNA to Protein (5 points)}
\textit{(Rosalind ID: PROT)}

Create \texttt{translate(rna)} using the codon table:
\begin{lstlisting}[style=codeStyle]
codon_table = {
    "UUU": "F", "UUC": "F", "UUA": "L", "UUG": "L",
    "UCU": "S", "UCC": "S", "UCA": "S", "UCG": "S",
    "UAU": "Y", "UAC": "Y", "UAA": "*", "UAG": "*",
    "UGU": "C", "UGC": "C", "UGA": "*", "UGG": "W",
    "CUU": "L", "CUC": "L", "CUA": "L", "CUG": "L",
    "CCU": "P", "CCC": "P", "CCA": "P", "CCG": "P",
    "CAU": "H", "CAC": "H", "CAA": "Q", "CAG": "Q",
    "CGU": "R", "CGC": "R", "CGA": "R", "CGG": "R",
    "AUU": "I", "AUC": "I", "AUA": "I", "AUG": "M",
    "ACU": "T", "ACC": "T", "ACA": "T", "ACG": "T",
    "AAU": "N", "AAC": "N", "AAA": "K", "AAG": "K",
    "AGU": "S", "AGC": "S", "AGA": "R", "AGG": "R",
    "GUU": "V", "GUC": "V", "GUA": "V", "GUG": "V",
    "GCU": "A", "GCC": "A", "GCA": "A", "GCG": "A",
    "GAU": "D", "GAC": "D", "GAA": "E", "GAG": "E",
    "GGU": "G", "GGC": "G", "GGA": "G", "GGG": "G"
}
\end{lstlisting}

\begin{enumerate}[label=\alph*)]
    \item Translate RNA to protein sequence
    \item Stop at stop codons (*, UAA, UAG, UGA)
    \item Handle sequences not divisible by 3
\end{enumerate}

% ============================================
% PART 2: CHEMINFORMATICS BASICS (25 points)
% ============================================
\section{Part 2: Cheminformatics - Molecules \& Properties (25 points)}

\subsection*{Problem 6: Molecular Formula Parser (5 points)}
Create \texttt{parse\_formula(formula)} that parses a molecular formula:
\begin{lstlisting}[style=codeStyle]
formulas = ["H2O", "C6H12O6", "C2H5OH", "NaCl", "Ca(OH)2"]
\end{lstlisting}

\begin{enumerate}[label=\alph*)]
    \item Return dictionary of element counts
    \item Handle parentheses (e.g., Ca(OH)2 = Ca:1, O:2, H:2)
    \item Handle subscripts correctly
\end{enumerate}

\subsection*{Problem 7: Molecular Weight Calculator (5 points)}
Create \texttt{calculate\_mw(formula)} using atomic weights:
\begin{lstlisting}[style=codeStyle]
atomic_weights = {
    "H": 1.008, "C": 12.011, "N": 14.007, "O": 15.999,
    "S": 32.065, "P": 30.974, "Na": 22.990, "Cl": 35.453,
    "Ca": 40.078, "Fe": 55.845, "Mg": 24.305, "K": 39.098
}
\end{lstlisting}

\begin{enumerate}[label=\alph*)]
    \item Calculate MW from parsed formula
    \item Round to 3 decimal places
    \item Test with: Aspirin (C9H8O4), Caffeine (C8H10N4O2), Glucose (C6H12O6)
\end{enumerate}

\subsection*{Problem 8: SMILES String Analyzer (5 points)}
Create functions to analyze SMILES strings (simplified, no RDKit):
\begin{lstlisting}[style=codeStyle]
smiles_list = [
    "CCO",           # Ethanol
    "CC(=O)O",       # Acetic acid
    "c1ccccc1",      # Benzene
    "CC(=O)Oc1ccccc1C(=O)O"  # Aspirin
]
\end{lstlisting}

\begin{enumerate}[label=\alph*)]
    \item \texttt{count\_atoms(smiles)}: Count C, N, O, S atoms (simplified)
    \item \texttt{has\_ring(smiles)}: Check if molecule has a ring (contains digits)
    \item \texttt{count\_double\_bonds(smiles)}: Count '=' occurrences
    \item \texttt{is\_aromatic(smiles)}: Check for lowercase letters (aromatic atoms)
\end{enumerate}

\subsection*{Problem 9: Drug-likeness Calculator (5 points)}
Implement Lipinski's Rule of Five checker:
\begin{lstlisting}[style=codeStyle]
# Drug properties dictionary
drugs = {
    "Aspirin": {"MW": 180.16, "LogP": 1.19, "HBD": 1, "HBA": 4},
    "Ibuprofen": {"MW": 206.29, "LogP": 3.97, "HBD": 1, "HBA": 2},
    "Metformin": {"MW": 129.16, "LogP": -1.43, "HBD": 3, "HBA": 5},
    "Atorvastatin": {"MW": 558.64, "LogP": 6.36, "HBD": 4, "HBA": 7}
}
\end{lstlisting}

\begin{enumerate}[label=\alph*)]
    \item \texttt{check\_lipinski(properties)}: Check all 5 rules
    \begin{itemize}
        \item MW $\leq$ 500
        \item LogP $\leq$ 5
        \item HBD (H-bond donors) $\leq$ 5
        \item HBA (H-bond acceptors) $\leq$ 10
    \end{itemize}
    \item Return number of violations
    \item \texttt{classify\_drugs(drugs\_dict)}: Classify all as ``Drug-like'' or ``Not Drug-like''
\end{enumerate}

\subsection*{Problem 10: IC50 to pIC50 Converter (5 points)}
Create functions for bioactivity data:
\begin{lstlisting}[style=codeStyle]
ic50_values = [0.5, 1.0, 10.0, 100.0, 1000.0, 5000.0]  # in nM
\end{lstlisting}

\begin{enumerate}[label=\alph*)]
    \item \texttt{ic50\_to\_pic50(ic50\_nm)}: Convert IC50 (nM) to pIC50
    \begin{itemize}
        \item pIC50 = -log10(IC50 in M) = 9 - log10(IC50 in nM)
    \end{itemize}
    \item \texttt{classify\_potency(pic50)}: Classify activity
    \begin{itemize}
        \item pIC50 $\geq$ 8: ``Highly Active''
        \item pIC50 $\geq$ 6: ``Active''
        \item pIC50 $\geq$ 5: ``Moderately Active''
        \item pIC50 $<$ 5: ``Inactive''
    \end{itemize}
    \item \texttt{analyze\_series(ic50\_list)}: Return statistics (min, max, mean pIC50)
\end{enumerate}

% ============================================
% PART 3: SEQUENCE ANALYSIS (25 points)
% ============================================
\section{Part 3: Advanced Sequence Analysis (25 points)}

\subsection*{Problem 11: Finding Motifs in DNA (5 points)}
\textit{(Rosalind ID: SUBS)}

Create \texttt{find\_motif(dna, motif)} that finds all occurrences:
\begin{lstlisting}[style=codeStyle]
dna = "GATATATGCATATACTT"
motif = "ATAT"
# Expected positions: [2, 4, 10] (1-indexed)
\end{lstlisting}

\begin{enumerate}[label=\alph*)]
    \item Return all starting positions (1-indexed, as in Rosalind)
    \item Handle overlapping matches
    \item Case-insensitive matching
\end{enumerate}

\subsection*{Problem 12: Hamming Distance (5 points)}
\textit{(Rosalind ID: HAMM)}

Create \texttt{hamming\_distance(s1, s2)} to count point mutations:
\begin{lstlisting}[style=codeStyle]
s1 = "GAGCCTACTAACGGGAT"
s2 = "CATCGTAATGACGGCCT"
# Expected: 7
\end{lstlisting}

\begin{enumerate}[label=\alph*)]
    \item Count positions where sequences differ
    \item Handle sequences of different lengths (raise error)
    \item Also implement \texttt{similarity\_percentage(s1, s2)}
\end{enumerate}

\subsection*{Problem 13: FASTA File Parser (5 points)}
\textit{(Rosalind ID: GC)}

Create \texttt{parse\_fasta(fasta\_string)} to parse FASTA format:
\begin{lstlisting}[style=codeStyle]
fasta_string = """>Rosalind_6404
CCTGCGGAAGATCGGCACTAGAATAGCCAGAACCGTTTCTCTGAGGCTTCCGGCCTTCCC
TCCCACTAATAATTCTGAGG
>Rosalind_5959
CCATCGGTAGCGCATCCTTAGTCCAATTAAGTCCCTATCCAGGCGCTCCGCCGAAGGTCT
ATATCCATTTGTCAGCAGACACGC"""
\end{lstlisting}

\begin{enumerate}[label=\alph*)]
    \item Return dictionary: \{sequence\_id: sequence\}
    \item Handle multi-line sequences
    \item Remove newlines from sequences
    \item Validate FASTA format
\end{enumerate}

\subsection*{Problem 14: Protein Mass Calculator (5 points)}
\textit{(Rosalind ID: PRTM)}

Calculate protein mass from amino acid sequence:
\begin{lstlisting}[style=codeStyle]
aa_mass = {
    "A": 71.04, "R": 156.10, "N": 114.04, "D": 115.03, "C": 103.01,
    "E": 129.04, "Q": 128.06, "G": 57.02, "H": 137.06, "I": 113.08,
    "L": 113.08, "K": 128.09, "M": 131.04, "F": 147.07, "P": 97.05,
    "S": 87.03, "T": 101.05, "W": 186.08, "Y": 163.06, "V": 99.07
}
\end{lstlisting}

\begin{enumerate}[label=\alph*)]
    \item \texttt{protein\_mass(sequence)}: Calculate total mass
    \item Handle invalid amino acids
    \item Return mass rounded to 3 decimal places
\end{enumerate}

Test with: \texttt{"SKADYEK"} (Expected: 821.392 Da)

\subsection*{Problem 15: Consensus and Profile (5 points)}
\textit{(Rosalind ID: CONS)}

Given multiple DNA strings of equal length, find the consensus:
\begin{lstlisting}[style=codeStyle]
sequences = [
    "ATCCAGCT",
    "GGGCAACT",
    "ATGGATCT",
    "AAGCAACC",
    "TTGGAACT",
    "ATGCCATT",
    "ATGGCACT"
]
\end{lstlisting}

\begin{enumerate}[label=\alph*)]
    \item \texttt{build\_profile(sequences)}: Create profile matrix (count of each nucleotide at each position)
    \item \texttt{consensus\_sequence(profile)}: Return consensus string
    \item Handle ties by choosing alphabetically first
\end{enumerate}

% ============================================
% PART 4: DATA ANALYSIS (25 points)
% ============================================
\section{Part 4: Scientific Data Analysis (25 points)}

\subsection*{Problem 16: Bioactivity Data Processing (5 points)}
Process a bioactivity dataset:
\begin{lstlisting}[style=codeStyle]
bioactivity_data = [
    {"compound": "CPD001", "target": "EGFR", "IC50_nM": 5.2, "MW": 423.5},
    {"compound": "CPD002", "target": "EGFR", "IC50_nM": 120.0, "MW": 389.2},
    {"compound": "CPD003", "target": "VEGFR", "IC50_nM": 8.7, "MW": 512.3},
    {"compound": "CPD004", "target": "EGFR", "IC50_nM": 2.1, "MW": 445.6},
    {"compound": "CPD005", "target": "VEGFR", "IC50_nM": 450.0, "MW": 378.9},
]
\end{lstlisting}

\begin{enumerate}[label=\alph*)]
    \item Add pIC50 values to each entry
    \item Filter compounds by target
    \item Find most potent compound per target
    \item Calculate average pIC50 per target
\end{enumerate}

\subsection*{Problem 17: Sequence Statistics with NumPy (5 points)}
Analyze multiple sequences using NumPy:
\begin{lstlisting}[style=codeStyle]
import numpy as np
sequences = ["ATGCGATCGATCG", "ATGCATGCATGCA", "GCGCGCGCGCGCG"]
\end{lstlisting}

\begin{enumerate}[label=\alph*)]
    \item Convert sequences to numerical encoding (A=0, C=1, G=2, T=3)
    \item Create a 2D NumPy array of encoded sequences
    \item Calculate nucleotide frequencies per position
    \item Find positions with highest variability
\end{enumerate}

\subsection*{Problem 18: Compound Library Analysis with Pandas (5 points)}
Create and analyze a compound library:
\begin{lstlisting}[style=codeStyle]
import pandas as pd
compounds = pd.DataFrame({
    "ID": ["CPD001", "CPD002", "CPD003", "CPD004", "CPD005"],
    "MW": [423.5, 389.2, 512.3, 445.6, 378.9],
    "LogP": [3.2, 2.8, 4.5, 3.8, 2.1],
    "HBD": [2, 1, 3, 2, 1],
    "HBA": [5, 4, 6, 5, 3],
    "IC50_nM": [5.2, 120.0, 8.7, 2.1, 450.0]
})
\end{lstlisting}

\begin{enumerate}[label=\alph*)]
    \item Add pIC50 column
    \item Add Lipinski\_Violations column
    \item Filter drug-like compounds (violations $\leq$ 1)
    \item Rank by potency (pIC50)
    \item Export filtered results to CSV
\end{enumerate}

\subsection*{Problem 19: Dose-Response Analysis (5 points)}
Analyze dose-response data:
\begin{lstlisting}[style=codeStyle]
doses = [0.001, 0.01, 0.1, 1, 10, 100, 1000]  # uM
responses = [2, 5, 15, 45, 78, 95, 99]  # % inhibition
\end{lstlisting}

\begin{enumerate}[label=\alph*)]
    \item Plot dose-response curve (log scale for dose)
    \item Estimate IC50 by interpolation (dose at 50\% response)
    \item Calculate Hill slope (steepness)
    \item Classify as: steep ($>$1.5), normal (0.8-1.5), shallow ($<$0.8)
\end{enumerate}

\subsection*{Problem 20: Molecular Descriptor Analysis (5 points)}
Analyze descriptor correlations:
\begin{lstlisting}[style=codeStyle]
descriptors = pd.DataFrame({
    "MW": [423, 389, 512, 445, 378, 520, 410, 395],
    "LogP": [3.2, 2.8, 4.5, 3.8, 2.1, 5.1, 3.0, 2.5],
    "TPSA": [78, 65, 95, 82, 55, 102, 70, 60],
    "RotBonds": [5, 4, 8, 6, 3, 9, 5, 4],
    "pIC50": [8.3, 6.9, 8.1, 8.7, 6.3, 5.8, 7.5, 7.0]
})
\end{lstlisting}

\begin{enumerate}[label=\alph*)]
    \item Calculate correlation matrix
    \item Find descriptors most correlated with pIC50
    \item Identify highly correlated descriptor pairs ($|r| > 0.7$)
    \item Create scatter plot of best predictor vs pIC50
    \item Simple linear regression to predict pIC50
\end{enumerate}

% ============================================
% SUBMISSION GUIDELINES
% ============================================
\section*{Submission Guidelines}

\begin{enumerate}
    \item Submit a single Python file: \texttt{homework\_solutions.py}
    \item Include all required imports at the top (numpy, pandas, math, re)
    \item Use comments to clearly label each problem
    \item Include docstrings for all functions
    \item Test all functions before submission
    \item Include sample output as comments where appropriate
\end{enumerate}

\subsection*{Resources}
\begin{itemize}
    \item Rosalind.info - Bioinformatics problems: \url{http://rosalind.info}
    \item NCBI - Nucleotide database: \url{https://www.ncbi.nlm.nih.gov}
    \item ChEMBL - Bioactivity database: \url{https://www.ebi.ac.uk/chembl/}
\end{itemize}

\subsection*{Grading Rubric}
\begin{itemize}
    \item \textbf{Correctness (60\%)}: Code produces correct output for test cases
    \item \textbf{Code Quality (20\%)}: Clean, readable, well-organized code
    \item \textbf{Documentation (10\%)}: Clear comments and docstrings
    \item \textbf{Error Handling (10\%)}: Appropriate validation and try/except
\end{itemize}

\vspace{1em}
\hrule
\vspace{1em}

\textbf{Due Date:} One week from assignment date\\
\textbf{Late Policy:} 10\% deduction per day late\\
\textbf{Questions:} Contact instructor via email or office hours

\end{document}

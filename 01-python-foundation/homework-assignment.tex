\documentclass{article}
\usepackage[utf8]{inputenc}
\usepackage{listings}
\usepackage{xcolor}
\usepackage{hyperref}
\usepackage{geometry}
\usepackage{enumitem}
\geometry{margin=1in}

% Python code style
\definecolor{codegray}{rgb}{0.95,0.95,0.95}
\definecolor{keyword}{rgb}{0.0,0.0,0.6}
\definecolor{comment}{rgb}{0.0,0.5,0.0}
\definecolor{string}{rgb}{0.64,0.08,0.08}

\lstdefinestyle{codeStyle}{
    language=Python,
    backgroundcolor=\color{codegray},
    commentstyle=\color{comment}\itshape,
    keywordstyle=\color{keyword}\bfseries,
    stringstyle=\color{string},
    basicstyle=\ttfamily\small,
    breaklines=true,
    frame=single,
    showstringspaces=false,
    tabsize=4,
    morekeywords={self, True, False, None, with, as, match, case}
}

\title{Python Programming: Comprehensive Homework Assignment}
\author{AI-Driven Development Training}
\date{February 2026}

\begin{document}

\maketitle

\section*{Instructions}
\begin{itemize}
    \item Complete all problems in a single Python file named \texttt{homework\_solutions.py}
    \item Use comments to separate each problem (e.g., \texttt{\# Problem 1})
    \item Include docstrings for all functions
    \item Test your code before submission
    \item Total Points: 100
    \item Estimated Time: 3-4 hours
\end{itemize}

\vspace{1em}
\hrule
\vspace{1em}

% ============================================
% PART 1: BASICS (25 points)
% ============================================
\section{Part 1: Python Basics (25 points)}

\subsection*{Problem 1: Student Information System (5 points)}
Create a program that:
\begin{enumerate}[label=\alph*)]
    \item Prompts the user for their name, age, and GPA (as float)
    \item Stores these in appropriate variables
    \item Prints a formatted message: ``Student [name], age [age], has a GPA of [GPA]''
    \item Determines and prints their academic standing:
    \begin{itemize}
        \item GPA $\geq$ 3.5: ``Dean's List''
        \item GPA $\geq$ 2.0: ``Good Standing''
        \item GPA $<$ 2.0: ``Academic Probation''
    \end{itemize}
\end{enumerate}

\subsection*{Problem 2: Shopping Cart Calculator (5 points)}
Write a program that calculates the total cost of items in a shopping cart:
\begin{enumerate}[label=\alph*)]
    \item Define a list of tuples: \texttt{cart = [("Apple", 3, 1.50), ("Bread", 2, 2.99), ("Milk", 1, 4.50)]}
    \item Each tuple contains: (item\_name, quantity, price\_per\_unit)
    \item Calculate the subtotal for each item
    \item Apply a 10\% discount if the total exceeds \$15
    \item Add 8\% sales tax to the final amount
    \item Print an itemized receipt with the final total
\end{enumerate}

\subsection*{Problem 3: Text Analyzer (5 points)}
Create a function \texttt{analyze\_text(text)} that takes a string and returns a dictionary with:
\begin{enumerate}[label=\alph*)]
    \item \texttt{"char\_count"}: Total characters (excluding spaces)
    \item \texttt{"word\_count"}: Total words
    \item \texttt{"sentence\_count"}: Total sentences (count periods, !, ?)
    \item \texttt{"avg\_word\_length"}: Average word length
    \item \texttt{"longest\_word"}: The longest word in the text
\end{enumerate}

Test with: \texttt{"Python is amazing! It is easy to learn. What do you think?"}

\subsection*{Problem 4: Number Guessing Game (5 points)}
Create a number guessing game:
\begin{enumerate}[label=\alph*)]
    \item Generate a random number between 1 and 100
    \item Allow the user up to 7 guesses
    \item After each guess, print ``Too high'' or ``Too low''
    \item If they guess correctly, print ``Congratulations!'' and how many attempts it took
    \item If they run out of guesses, reveal the number
    \item Handle invalid input gracefully using try/except
\end{enumerate}

\subsection*{Problem 5: Temperature Statistics (5 points)}
Given a week of temperature readings:
\begin{lstlisting}[style=codeStyle]
temperatures = [72, 68, 75, 80, 82, 78, 71]  # Fahrenheit
\end{lstlisting}

Write functions to:
\begin{enumerate}[label=\alph*)]
    \item \texttt{convert\_all\_to\_celsius(temps)}: Convert all to Celsius
    \item \texttt{get\_statistics(temps)}: Return dict with min, max, average
    \item \texttt{categorize\_temps(temps)}: Return list of categories (``Cold'' $<$ 70, ``Comfortable'' 70-80, ``Hot'' $>$ 80)
\end{enumerate}

% ============================================
% PART 2: COLLECTIONS (25 points)
% ============================================
\section{Part 2: Collections \& Data Structures (25 points)}

\subsection*{Problem 6: Inventory Management (5 points)}
Create an inventory system using dictionaries:
\begin{lstlisting}[style=codeStyle]
inventory = {
    "laptop": {"price": 999.99, "quantity": 15, "category": "electronics"},
    "shirt": {"price": 29.99, "quantity": 50, "category": "clothing"},
    "book": {"price": 15.99, "quantity": 100, "category": "media"},
    "headphones": {"price": 149.99, "quantity": 30, "category": "electronics"}
}
\end{lstlisting}

Implement:
\begin{enumerate}[label=\alph*)]
    \item \texttt{get\_total\_value()}: Calculate total inventory value (price $\times$ quantity for all items)
    \item \texttt{get\_items\_by\_category(category)}: Return list of items in a category
    \item \texttt{update\_quantity(item, change)}: Update quantity (positive to add, negative to remove)
    \item \texttt{low\_stock\_alert(threshold)}: Return items with quantity below threshold
\end{enumerate}

\subsection*{Problem 7: Student Gradebook (5 points)}
Create a gradebook system:
\begin{lstlisting}[style=codeStyle]
gradebook = {
    "Alice": {"math": 85, "science": 92, "english": 88},
    "Bob": {"math": 78, "science": 85, "english": 90},
    "Charlie": {"math": 92, "science": 88, "english": 85},
    "Diana": {"math": 95, "science": 98, "english": 92}
}
\end{lstlisting}

Implement:
\begin{enumerate}[label=\alph*)]
    \item \texttt{calculate\_gpa(student)}: Calculate average grade for a student
    \item \texttt{class\_average(subject)}: Calculate class average for a subject
    \item \texttt{honor\_roll()}: Return students with GPA $\geq$ 90
    \item \texttt{subject\_rankings(subject)}: Return students sorted by grade in that subject (highest first)
\end{enumerate}

\subsection*{Problem 8: List Operations Challenge (5 points)}
Given two lists:
\begin{lstlisting}[style=codeStyle]
list_a = [1, 2, 3, 4, 5, 6, 7, 8, 9, 10]
list_b = [5, 10, 15, 20, 25]
\end{lstlisting}

Using list comprehensions and set operations:
\begin{enumerate}[label=\alph*)]
    \item Create a list of squares of even numbers from list\_a
    \item Find elements common to both lists
    \item Find elements in list\_a but not in list\_b
    \item Create a list of tuples: (number, square, cube) for numbers 1-10
    \item Flatten a nested list: \texttt{[[1,2], [3,4], [5,6]]} into \texttt{[1,2,3,4,5,6]}
\end{enumerate}

\subsection*{Problem 9: Word Frequency Counter (5 points)}
Create a program that analyzes text and finds word frequencies:
\begin{lstlisting}[style=codeStyle]
text = """Python is a popular programming language. Python is easy to learn.
Programming in Python is fun. Many developers love Python programming."""
\end{lstlisting}

Implement:
\begin{enumerate}[label=\alph*)]
    \item \texttt{word\_frequency(text)}: Return dict of word counts (case-insensitive)
    \item \texttt{top\_n\_words(freq\_dict, n)}: Return top n most frequent words
    \item \texttt{unique\_words(text)}: Return set of unique words
    \item Remove common stop words: ``is'', ``a'', ``to'', ``in'', ``the''
\end{enumerate}

\subsection*{Problem 10: Tuple Operations (5 points)}
Work with a list of student records (tuples):
\begin{lstlisting}[style=codeStyle]
students = [
    ("Alice", 22, "CS", 3.8),
    ("Bob", 20, "Math", 3.5),
    ("Charlie", 21, "CS", 3.9),
    ("Diana", 23, "Physics", 3.7),
    ("Eve", 20, "Math", 3.6)
]
# Format: (name, age, major, gpa)
\end{lstlisting}

Implement using lambda and sorted/filter:
\begin{enumerate}[label=\alph*)]
    \item Sort students by GPA (highest first)
    \item Filter students in ``CS'' major
    \item Find the youngest student
    \item Group students by major (return dict)
    \item Calculate average GPA per major
\end{enumerate}

% ============================================
% PART 3: FILE HANDLING & DATA (25 points)
% ============================================
\section{Part 3: File Handling \& Data Processing (25 points)}

\subsection*{Problem 11: Log File Processor (5 points)}
Create a log file processor that:
\begin{enumerate}[label=\alph*)]
    \item Creates a sample log file with 20 entries mixing INFO, WARNING, ERROR levels
    \item Reads the log file and categorizes entries by level
    \item Counts occurrences of each log level
    \item Writes only ERROR entries to a separate file ``errors.txt''
    \item Prints a summary report
\end{enumerate}

Sample log format: \texttt{[LEVEL] 2026-02-11 10:30:00 - Message here}

\subsection*{Problem 12: CSV Data Handler (5 points)}
Create functions to work with CSV-like data (without using the csv module):
\begin{enumerate}[label=\alph*)]
    \item \texttt{create\_csv(filename, headers, data)}: Create a CSV file with headers and data
    \item \texttt{read\_csv(filename)}: Read CSV and return list of dictionaries
    \item \texttt{filter\_csv(data, column, value)}: Filter rows where column equals value
    \item \texttt{add\_row(filename, row\_data)}: Append a new row to existing CSV
\end{enumerate}

Test with employee data: Name, Department, Salary, Years

\subsection*{Problem 13: JSON Configuration Manager (5 points)}
Create a configuration manager using JSON:
\begin{enumerate}[label=\alph*)]
    \item \texttt{create\_config(filename)}: Create default config with app settings
    \item \texttt{load\_config(filename)}: Load config from file
    \item \texttt{update\_config(filename, key, value)}: Update a specific setting
    \item \texttt{validate\_config(config)}: Check all required keys exist
\end{enumerate}

Required settings: ``app\_name'', ``version'', ``debug\_mode'', ``max\_users'', ``database\_url''

\subsection*{Problem 14: Data Validation with Regex (5 points)}
Create validation functions using regex:
\begin{enumerate}[label=\alph*)]
    \item \texttt{validate\_email(email)}: Check valid email format
    \item \texttt{validate\_phone(phone)}: Check format: (XXX) XXX-XXXX or XXX-XXX-XXXX
    \item \texttt{validate\_password(password)}: At least 8 chars, 1 uppercase, 1 lowercase, 1 digit
    \item \texttt{extract\_dates(text)}: Find all dates in format MM/DD/YYYY or YYYY-MM-DD
    \item \texttt{sanitize\_input(text)}: Remove special characters except letters, numbers, spaces
\end{enumerate}

\subsection*{Problem 15: Error Handling System (5 points)}
Create a robust data processor with comprehensive error handling:
\begin{enumerate}[label=\alph*)]
    \item \texttt{safe\_divide(a, b)}: Handle ZeroDivisionError, TypeError
    \item \texttt{safe\_file\_read(filename)}: Handle FileNotFoundError, PermissionError
    \item \texttt{safe\_json\_parse(json\_string)}: Handle JSONDecodeError
    \item \texttt{safe\_list\_access(lst, index)}: Handle IndexError
    \item Create a decorator \texttt{@log\_errors} that logs any exception to ``error\_log.txt''
\end{enumerate}

% ============================================
% PART 4: NUMPY & PANDAS (25 points)
% ============================================
\section{Part 4: NumPy \& Pandas (25 points)}

\subsection*{Problem 16: NumPy Array Operations (5 points)}
Using NumPy, perform the following operations:
\begin{enumerate}[label=\alph*)]
    \item Create a 5x5 matrix with random integers between 1 and 100
    \item Calculate the sum, mean, and standard deviation of each row
    \item Find the maximum value and its position in the matrix
    \item Replace all values greater than 50 with 50 (capping)
    \item Transpose the matrix and multiply it with the original
\end{enumerate}

\subsection*{Problem 17: NumPy Statistical Analysis (5 points)}
Given sales data as a NumPy array:
\begin{lstlisting}[style=codeStyle]
# Monthly sales for 4 products over 12 months
sales = np.random.randint(100, 1000, size=(4, 12))
\end{lstlisting}

Calculate:
\begin{enumerate}[label=\alph*)]
    \item Total sales per product (row sums)
    \item Total sales per month (column sums)
    \item Best performing product and month
    \item Products with above-average total sales
    \item Month-over-month percentage change for each product
\end{enumerate}

\subsection*{Problem 18: Pandas DataFrame Creation (5 points)}
Create a comprehensive employee DataFrame:
\begin{enumerate}[label=\alph*)]
    \item Create DataFrame with: ID, Name, Department, Salary, Hire\_Date, Performance\_Score
    \item Add at least 10 employees with realistic data
    \item Set the ID column as the index
    \item Add a new column ``Years\_Employed'' calculated from Hire\_Date
    \item Add a ``Bonus'' column: 15\% of salary for score $\geq$ 8, 10\% for score $\geq$ 6, else 5\%
\end{enumerate}

\subsection*{Problem 19: Pandas Data Analysis (5 points)}
Using the employee DataFrame from Problem 18:
\begin{enumerate}[label=\alph*)]
    \item Find the average salary by department
    \item Find the top 3 highest-paid employees
    \item Count employees per department
    \item Find employees hired in the last 2 years
    \item Calculate the correlation between Years\_Employed and Performance\_Score
\end{enumerate}

\subsection*{Problem 20: Pandas Advanced Operations (5 points)}
Create a sales analysis system:
\begin{lstlisting}[style=codeStyle]
# Create sales DataFrame
sales_data = {
    'Date': pd.date_range('2025-01-01', periods=100),
    'Product': np.random.choice(['A', 'B', 'C', 'D'], 100),
    'Region': np.random.choice(['North', 'South', 'East', 'West'], 100),
    'Quantity': np.random.randint(1, 50, 100),
    'Unit_Price': np.random.uniform(10, 100, 100).round(2)
}
\end{lstlisting}

Implement:
\begin{enumerate}[label=\alph*)]
    \item Add ``Revenue'' column (Quantity $\times$ Unit\_Price)
    \item Create pivot table: Products vs Regions with Revenue
    \item Group by month and calculate monthly totals
    \item Find the best-selling product in each region
    \item Export results to ``sales\_summary.csv''
\end{enumerate}

% ============================================
% SUBMISSION GUIDELINES
% ============================================
\section*{Submission Guidelines}

\begin{enumerate}
    \item Submit a single Python file: \texttt{homework\_solutions.py}
    \item Include all required imports at the top
    \item Use comments to clearly label each problem
    \item Include docstrings for all functions
    \item Test all functions before submission
    \item Include sample output as comments where appropriate
\end{enumerate}

\subsection*{Grading Rubric}
\begin{itemize}
    \item \textbf{Correctness (60\%)}: Code produces correct output
    \item \textbf{Code Quality (20\%)}: Clean, readable, well-organized code
    \item \textbf{Documentation (10\%)}: Clear comments and docstrings
    \item \textbf{Error Handling (10\%)}: Appropriate use of try/except
\end{itemize}

\vspace{1em}
\hrule
\vspace{1em}

\textbf{Due Date:} One week from assignment date\\
\textbf{Late Policy:} 10\% deduction per day late\\
\textbf{Questions:} Contact instructor via email or office hours

\end{document}

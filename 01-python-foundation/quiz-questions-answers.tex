\documentclass{article}
\usepackage[utf8]{inputenc}
\usepackage{listings}
\usepackage{xcolor}
\usepackage{hyperref}
\usepackage{geometry}
\usepackage{enumitem}
\usepackage{longtable}
\geometry{margin=0.75in}

% Python code style
\definecolor{codegray}{rgb}{0.95,0.95,0.95}
\definecolor{keyword}{rgb}{0.0,0.0,0.6}
\definecolor{comment}{rgb}{0.0,0.5,0.0}
\definecolor{string}{rgb}{0.64,0.08,0.08}
\definecolor{correct}{rgb}{0.0,0.5,0.0}

\lstdefinestyle{codeStyle}{
    language=Python,
    backgroundcolor=\color{codegray},
    commentstyle=\color{comment}\itshape,
    keywordstyle=\color{keyword}\bfseries,
    stringstyle=\color{string},
    basicstyle=\ttfamily\small,
    breaklines=true,
    frame=single,
    showstringspaces=false,
    tabsize=4,
    morekeywords={self, True, False, None, with, as, match, case}
}

\title{Python Programming: 50 Quiz Questions\\with Answers and Rationales}
\author{AI-Driven Development Training}
\date{February 2026}

\begin{document}

\maketitle
\tableofcontents
\newpage

% ============================================
% SECTION 1: VARIABLES & DATA TYPES (Q1-Q5)
% ============================================
\section{Variables \& Data Types (Questions 1-5)}

\subsection*{Question 1}
What is the output of the following code?
\begin{lstlisting}[style=codeStyle]
x = 5
y = "5"
print(type(x) == type(y))
\end{lstlisting}

\textbf{Options:}
\begin{enumerate}[label=\Alph*)]
    \item True
    \item False
    \item Error
    \item None
\end{enumerate}

\textbf{Answer: B) False}

\textbf{Rationale:} \texttt{x} is an integer (\texttt{int}) while \texttt{y} is a string (\texttt{str}). The \texttt{type()} function returns the data type, and comparing \texttt{<class 'int'>} with \texttt{<class 'str'>} returns \texttt{False}.

\hrule
\vspace{1em}

\subsection*{Question 2}
What will be the value of \texttt{result}?
\begin{lstlisting}[style=codeStyle]
a, b, c = 1, 2, 3
result = a, b = b, a
print(b)
\end{lstlisting}

\textbf{Options:}
\begin{enumerate}[label=\Alph*)]
    \item 1
    \item 2
    \item (2, 1)
    \item Error
\end{enumerate}

\textbf{Answer: A) 1}

\textbf{Rationale:} The statement \texttt{a, b = b, a} performs tuple unpacking and swaps the values. After execution, \texttt{a = 2} and \texttt{b = 1}. The \texttt{result} variable holds the tuple \texttt{(2, 1)}.

\hrule
\vspace{1em}

\subsection*{Question 3}
What is the output?
\begin{lstlisting}[style=codeStyle]
x = None
print(bool(x), bool(0), bool(""), bool([]))
\end{lstlisting}

\textbf{Options:}
\begin{enumerate}[label=\Alph*)]
    \item True True True True
    \item False False False False
    \item None 0 "" []
    \item Error
\end{enumerate}

\textbf{Answer: B) False False False False}

\textbf{Rationale:} In Python, \texttt{None}, \texttt{0}, empty strings \texttt{""}, and empty lists \texttt{[]} are all ``falsy'' values. When converted to boolean using \texttt{bool()}, they all return \texttt{False}.

\hrule
\vspace{1em}

\subsection*{Question 4}
What is the result of \texttt{int(3.9) + int(-3.9)}?

\textbf{Options:}
\begin{enumerate}[label=\Alph*)]
    \item 0
    \item 1
    \item -1
    \item 7
\end{enumerate}

\textbf{Answer: A) 0}

\textbf{Rationale:} \texttt{int()} truncates towards zero. \texttt{int(3.9) = 3} and \texttt{int(-3.9) = -3}. Therefore, \texttt{3 + (-3) = 0}.

\hrule
\vspace{1em}

\subsection*{Question 5}
Which statement correctly creates a multi-line string?

\textbf{Options:}
\begin{enumerate}[label=\Alph*)]
    \item \texttt{s = "Hello\textbackslash nWorld"}
    \item \texttt{s = '''Hello World'''}
    \item \texttt{s = """Hello\textbackslash nWorld"""}
    \item All of the above
\end{enumerate}

\textbf{Answer: D) All of the above}

\textbf{Rationale:} Multi-line strings can be created using: escape character \texttt{\textbackslash n}, triple single quotes \texttt{'''}, or triple double quotes \texttt{"""}. All methods are valid.

\hrule
\vspace{1em}

% ============================================
% SECTION 2: OPERATORS (Q6-Q10)
% ============================================
\section{Operators (Questions 6-10)}

\subsection*{Question 6}
What is the output?
\begin{lstlisting}[style=codeStyle]
print(17 // 5, 17 % 5, 17 / 5)
\end{lstlisting}

\textbf{Options:}
\begin{enumerate}[label=\Alph*)]
    \item 3 2 3.4
    \item 3.4 2 3
    \item 3 2 3.0
    \item 3.0 2.0 3.4
\end{enumerate}

\textbf{Answer: A) 3 2 3.4}

\textbf{Rationale:} \texttt{//} is floor division (returns integer quotient), \texttt{\%} is modulo (remainder), and \texttt{/} is true division (returns float). 17//5=3, 17\%5=2, 17/5=3.4.

\hrule
\vspace{1em}

\subsection*{Question 7}
What is the output?
\begin{lstlisting}[style=codeStyle]
x = 10
print(x > 5 and x < 15 or x == 20)
\end{lstlisting}

\textbf{Options:}
\begin{enumerate}[label=\Alph*)]
    \item True
    \item False
    \item 10
    \item Error
\end{enumerate}

\textbf{Answer: A) True}

\textbf{Rationale:} \texttt{and} has higher precedence than \texttt{or}. First, \texttt{x > 5 and x < 15} evaluates to \texttt{True and True = True}. Then \texttt{True or False = True}.

\hrule
\vspace{1em}

\subsection*{Question 8}
What is the value of \texttt{2 ** 3 ** 2}?

\textbf{Options:}
\begin{enumerate}[label=\Alph*)]
    \item 64
    \item 512
    \item 256
    \item 81
\end{enumerate}

\textbf{Answer: B) 512}

\textbf{Rationale:} The \texttt{**} operator is right-associative. So \texttt{2 ** 3 ** 2} = \texttt{2 ** (3 ** 2)} = \texttt{2 ** 9} = 512.

\hrule
\vspace{1em}

\subsection*{Question 9}
What does \texttt{not not True} evaluate to?

\textbf{Options:}
\begin{enumerate}[label=\Alph*)]
    \item True
    \item False
    \item None
    \item Error
\end{enumerate}

\textbf{Answer: A) True}

\textbf{Rationale:} \texttt{not True = False}, then \texttt{not False = True}. Double negation returns the original value.

\hrule
\vspace{1em}

\subsection*{Question 10}
What is the output?
\begin{lstlisting}[style=codeStyle]
a = [1, 2, 3]
b = [1, 2, 3]
print(a == b, a is b)
\end{lstlisting}

\textbf{Options:}
\begin{enumerate}[label=\Alph*)]
    \item True True
    \item True False
    \item False True
    \item False False
\end{enumerate}

\textbf{Answer: B) True False}

\textbf{Rationale:} \texttt{==} compares values (both lists have same elements), \texttt{is} compares identity (they are different objects in memory). Same content, different objects.

\hrule
\vspace{1em}

% ============================================
% SECTION 3: STRINGS (Q11-Q15)
% ============================================
\section{Strings (Questions 11-15)}

\subsection*{Question 11}
What is the output?
\begin{lstlisting}[style=codeStyle]
s = "Python"
print(s[1:4], s[-3:], s[::2])
\end{lstlisting}

\textbf{Options:}
\begin{enumerate}[label=\Alph*)]
    \item yth hon Pto
    \item yth hon Pto
    \item Pyt hon Pto
    \item yth tho Pto
\end{enumerate}

\textbf{Answer: A) yth hon Pto}

\textbf{Rationale:} \texttt{s[1:4]} = characters at index 1,2,3 = ``yth''. \texttt{s[-3:]} = last 3 characters = ``hon''. \texttt{s[::2]} = every 2nd character = ``Pto''.

\hrule
\vspace{1em}

\subsection*{Question 12}
What is the output?
\begin{lstlisting}[style=codeStyle]
print("Hello" * 2 + "World")
\end{lstlisting}

\textbf{Options:}
\begin{enumerate}[label=\Alph*)]
    \item HelloHelloWorld
    \item Hello2World
    \item HelloWorldWorld
    \item Error
\end{enumerate}

\textbf{Answer: A) HelloHelloWorld}

\textbf{Rationale:} String multiplication \texttt{* 2} repeats the string twice, then \texttt{+} concatenates with ``World''.

\hrule
\vspace{1em}

\subsection*{Question 13}
What method removes whitespace from both ends of a string?

\textbf{Options:}
\begin{enumerate}[label=\Alph*)]
    \item \texttt{trim()}
    \item \texttt{strip()}
    \item \texttt{clean()}
    \item \texttt{remove()}
\end{enumerate}

\textbf{Answer: B) strip()}

\textbf{Rationale:} \texttt{strip()} removes leading and trailing whitespace. \texttt{lstrip()} removes left only, \texttt{rstrip()} removes right only. Python doesn't have \texttt{trim()}.

\hrule
\vspace{1em}

\subsection*{Question 14}
What is the output?
\begin{lstlisting}[style=codeStyle]
s = "a,b,c,d"
print(s.split(",")[2])
\end{lstlisting}

\textbf{Options:}
\begin{enumerate}[label=\Alph*)]
    \item a
    \item b
    \item c
    \item d
\end{enumerate}

\textbf{Answer: C) c}

\textbf{Rationale:} \texttt{split(",")} creates list \texttt{['a', 'b', 'c', 'd']}. Index 2 is the third element, which is ``c''.

\hrule
\vspace{1em}

\subsection*{Question 15}
What is the output?
\begin{lstlisting}[style=codeStyle]
name = "Alice"
age = 25
print(f"{name} is {age} years old")
\end{lstlisting}

\textbf{Options:}
\begin{enumerate}[label=\Alph*)]
    \item \{name\} is \{age\} years old
    \item Alice is 25 years old
    \item name is age years old
    \item Error
\end{enumerate}

\textbf{Answer: B) Alice is 25 years old}

\textbf{Rationale:} f-strings (formatted string literals) replace expressions inside \texttt{\{\}} with their values. \texttt{\{name\}} becomes ``Alice'' and \texttt{\{age\}} becomes 25.

\hrule
\vspace{1em}

% ============================================
% SECTION 4: CONDITIONALS (Q16-Q20)
% ============================================
\section{Conditionals (Questions 16-20)}

\subsection*{Question 16}
What is the output?
\begin{lstlisting}[style=codeStyle]
x = 5
if x > 3:
    if x > 7:
        print("A")
    else:
        print("B")
else:
    print("C")
\end{lstlisting}

\textbf{Options:}
\begin{enumerate}[label=\Alph*)]
    \item A
    \item B
    \item C
    \item No output
\end{enumerate}

\textbf{Answer: B) B}

\textbf{Rationale:} \texttt{x=5 > 3} is True, so we enter first if. Then \texttt{5 > 7} is False, so we go to else and print ``B''.

\hrule
\vspace{1em}

\subsection*{Question 17}
What is the output?
\begin{lstlisting}[style=codeStyle]
score = 75
grade = "A" if score >= 90 else "B" if score >= 80 else "C"
print(grade)
\end{lstlisting}

\textbf{Options:}
\begin{enumerate}[label=\Alph*)]
    \item A
    \item B
    \item C
    \item Error
\end{enumerate}

\textbf{Answer: C) C}

\textbf{Rationale:} This is a chained ternary operator. 75 >= 90 is False, 75 >= 80 is False, so grade = ``C''.

\hrule
\vspace{1em}

\subsection*{Question 18}
Which value is NOT considered ``falsy'' in Python?

\textbf{Options:}
\begin{enumerate}[label=\Alph*)]
    \item 0
    \item ""
    \item []
    \item "False"
\end{enumerate}

\textbf{Answer: D) "False"}

\textbf{Rationale:} The string \texttt{"False"} is a non-empty string, which is truthy. Only empty strings are falsy. The word ``False'' inside doesn't make it boolean False.

\hrule
\vspace{1em}

\subsection*{Question 19}
What is the output? (Python 3.10+)
\begin{lstlisting}[style=codeStyle]
status = 404
match status:
    case 200:
        print("OK")
    case 404:
        print("Not Found")
    case _:
        print("Unknown")
\end{lstlisting}

\textbf{Options:}
\begin{enumerate}[label=\Alph*)]
    \item OK
    \item Not Found
    \item Unknown
    \item Error
\end{enumerate}

\textbf{Answer: B) Not Found}

\textbf{Rationale:} The \texttt{match-case} statement matches \texttt{status=404} with \texttt{case 404}, printing ``Not Found''. The underscore \texttt{\_} is the wildcard/default case.

\hrule
\vspace{1em}

\subsection*{Question 20}
What is the output?
\begin{lstlisting}[style=codeStyle]
x = 10
y = 20
print("Equal" if x == y else "X > Y" if x > y else "X < Y")
\end{lstlisting}

\textbf{Options:}
\begin{enumerate}[label=\Alph*)]
    \item Equal
    \item X > Y
    \item X < Y
    \item Error
\end{enumerate}

\textbf{Answer: C) X < Y}

\textbf{Rationale:} \texttt{x == y} (10==20) is False. \texttt{x > y} (10>20) is False. So the final else returns ``X < Y''.

\hrule
\vspace{1em}

% ============================================
% SECTION 5: LOOPS (Q21-Q25)
% ============================================
\section{Loops (Questions 21-25)}

\subsection*{Question 21}
What is the output?
\begin{lstlisting}[style=codeStyle]
for i in range(3):
    print(i, end=" ")
\end{lstlisting}

\textbf{Options:}
\begin{enumerate}[label=\Alph*)]
    \item 1 2 3
    \item 0 1 2
    \item 0 1 2 3
    \item 1 2
\end{enumerate}

\textbf{Answer: B) 0 1 2}

\textbf{Rationale:} \texttt{range(3)} generates 0, 1, 2 (not including 3). The \texttt{end=" "} keeps output on same line with spaces.

\hrule
\vspace{1em}

\subsection*{Question 22}
What is the output?
\begin{lstlisting}[style=codeStyle]
for i in range(5):
    if i == 3:
        continue
    print(i, end=" ")
\end{lstlisting}

\textbf{Options:}
\begin{enumerate}[label=\Alph*)]
    \item 0 1 2 3 4
    \item 0 1 2 4
    \item 0 1 2
    \item 3
\end{enumerate}

\textbf{Answer: B) 0 1 2 4}

\textbf{Rationale:} \texttt{continue} skips the current iteration when \texttt{i == 3}, so 3 is not printed. All other values (0, 1, 2, 4) are printed.

\hrule
\vspace{1em}

\subsection*{Question 23}
What is the output?
\begin{lstlisting}[style=codeStyle]
i = 0
while i < 5:
    i += 1
    if i == 3:
        break
print(i)
\end{lstlisting}

\textbf{Options:}
\begin{enumerate}[label=\Alph*)]
    \item 2
    \item 3
    \item 4
    \item 5
\end{enumerate}

\textbf{Answer: B) 3}

\textbf{Rationale:} The loop increments \texttt{i} first, then checks if \texttt{i == 3}. When \texttt{i} becomes 3, \texttt{break} exits the loop, and \texttt{i} remains 3.

\hrule
\vspace{1em}

\subsection*{Question 24}
What is the output?
\begin{lstlisting}[style=codeStyle]
for i in range(3):
    for j in range(2):
        print(f"{i},{j}", end=" ")
\end{lstlisting}

\textbf{Options:}
\begin{enumerate}[label=\Alph*)]
    \item 0,0 0,1 1,0 1,1 2,0 2,1
    \item 0,0 1,1 2,2
    \item 0,0 0,1 0,2 1,0 1,1 1,2
    \item Error
\end{enumerate}

\textbf{Answer: A) 0,0 0,1 1,0 1,1 2,0 2,1}

\textbf{Rationale:} Nested loops: outer loop runs 3 times (i=0,1,2), inner loop runs 2 times for each outer iteration (j=0,1). Total 6 combinations.

\hrule
\vspace{1em}

\subsection*{Question 25}
What is the output?
\begin{lstlisting}[style=codeStyle]
for i in range(5):
    pass
else:
    print("Done")
\end{lstlisting}

\textbf{Options:}
\begin{enumerate}[label=\Alph*)]
    \item Nothing
    \item Done
    \item Error
    \item 0 1 2 3 4 Done
\end{enumerate}

\textbf{Answer: B) Done}

\textbf{Rationale:} The \texttt{else} clause of a \texttt{for} loop executes when the loop completes normally (not broken). \texttt{pass} does nothing, loop finishes normally, ``Done'' prints.

\hrule
\vspace{1em}

% ============================================
% SECTION 6: FUNCTIONS (Q26-Q30)
% ============================================
\section{Functions (Questions 26-30)}

\subsection*{Question 26}
What is the output?
\begin{lstlisting}[style=codeStyle]
def greet(name="World"):
    return f"Hello, {name}!"

print(greet())
\end{lstlisting}

\textbf{Options:}
\begin{enumerate}[label=\Alph*)]
    \item Hello, !
    \item Hello, World!
    \item Error
    \item None
\end{enumerate}

\textbf{Answer: B) Hello, World!}

\textbf{Rationale:} When no argument is passed, the default value ``World'' is used for the \texttt{name} parameter.

\hrule
\vspace{1em}

\subsection*{Question 27}
What is the output?
\begin{lstlisting}[style=codeStyle]
def add(*args):
    return sum(args)

print(add(1, 2, 3, 4, 5))
\end{lstlisting}

\textbf{Options:}
\begin{enumerate}[label=\Alph*)]
    \item (1, 2, 3, 4, 5)
    \item 15
    \item [1, 2, 3, 4, 5]
    \item Error
\end{enumerate}

\textbf{Answer: B) 15}

\textbf{Rationale:} \texttt{*args} collects all positional arguments into a tuple. \texttt{sum((1,2,3,4,5))} = 15.

\hrule
\vspace{1em}

\subsection*{Question 28}
What is the output?
\begin{lstlisting}[style=codeStyle]
def info(**kwargs):
    return len(kwargs)

print(info(name="Alice", age=25, city="NYC"))
\end{lstlisting}

\textbf{Options:}
\begin{enumerate}[label=\Alph*)]
    \item 3
    \item \{name: Alice, age: 25, city: NYC\}
    \item Error
    \item None
\end{enumerate}

\textbf{Answer: A) 3}

\textbf{Rationale:} \texttt{**kwargs} collects keyword arguments into a dictionary. Three key-value pairs are passed, so \texttt{len(kwargs) = 3}.

\hrule
\vspace{1em}

\subsection*{Question 29}
What is the output?
\begin{lstlisting}[style=codeStyle]
def outer():
    x = 10
    def inner():
        return x * 2
    return inner

f = outer()
print(f())
\end{lstlisting}

\textbf{Options:}
\begin{enumerate}[label=\Alph*)]
    \item 10
    \item 20
    \item Error
    \item None
\end{enumerate}

\textbf{Answer: B) 20}

\textbf{Rationale:} This demonstrates closure. \texttt{outer()} returns \texttt{inner} function which remembers \texttt{x=10} from enclosing scope. Calling \texttt{f()} returns \texttt{10 * 2 = 20}.

\hrule
\vspace{1em}

\subsection*{Question 30}
What is the output?
\begin{lstlisting}[style=codeStyle]
def func(a, b=2, *args, **kwargs):
    return a + b + sum(args) + len(kwargs)

print(func(1, 3, 4, 5, x=10, y=20))
\end{lstlisting}

\textbf{Options:}
\begin{enumerate}[label=\Alph*)]
    \item 15
    \item 13
    \item 43
    \item Error
\end{enumerate}

\textbf{Answer: A) 15}

\textbf{Rationale:} \texttt{a=1}, \texttt{b=3}, \texttt{args=(4,5)}, \texttt{kwargs=\{x:10, y:20\}}. Result: \texttt{1 + 3 + (4+5) + 2 = 15}.

\hrule
\vspace{1em}

% ============================================
% SECTION 7: ERROR HANDLING (Q31-Q33)
% ============================================
\section{Error Handling (Questions 31-33)}

\subsection*{Question 31}
What is the output?
\begin{lstlisting}[style=codeStyle]
try:
    x = 10 / 0
except ZeroDivisionError:
    print("A")
except:
    print("B")
else:
    print("C")
finally:
    print("D")
\end{lstlisting}

\textbf{Options:}
\begin{enumerate}[label=\Alph*)]
    \item A D
    \item B D
    \item A C D
    \item C D
\end{enumerate}

\textbf{Answer: A) A D}

\textbf{Rationale:} Division by zero raises \texttt{ZeroDivisionError}, caught by first except (prints ``A''). \texttt{else} runs only if no exception. \texttt{finally} always runs (prints ``D'').

\hrule
\vspace{1em}

\subsection*{Question 32}
Which exception is raised when accessing a non-existent dictionary key?

\textbf{Options:}
\begin{enumerate}[label=\Alph*)]
    \item IndexError
    \item KeyError
    \item ValueError
    \item AttributeError
\end{enumerate}

\textbf{Answer: B) KeyError}

\textbf{Rationale:} \texttt{KeyError} is raised when a dictionary key is not found. \texttt{IndexError} is for list indices, \texttt{ValueError} for invalid values, \texttt{AttributeError} for missing attributes.

\hrule
\vspace{1em}

\subsection*{Question 33}
What is the output?
\begin{lstlisting}[style=codeStyle]
def risky():
    try:
        return 1
    finally:
        return 2

print(risky())
\end{lstlisting}

\textbf{Options:}
\begin{enumerate}[label=\Alph*)]
    \item 1
    \item 2
    \item 1 2
    \item Error
\end{enumerate}

\textbf{Answer: B) 2}

\textbf{Rationale:} The \texttt{finally} block always executes, even after a return. The return in \texttt{finally} overrides the return in \texttt{try}, so 2 is returned.

\hrule
\vspace{1em}

% ============================================
% SECTION 8: LISTS (Q34-Q38)
% ============================================
\section{Lists (Questions 34-38)}

\subsection*{Question 34}
What is the output?
\begin{lstlisting}[style=codeStyle]
lst = [1, 2, 3, 4, 5]
print(lst[1:4])
\end{lstlisting}

\textbf{Options:}
\begin{enumerate}[label=\Alph*)]
    \item [1, 2, 3, 4]
    \item [2, 3, 4]
    \item [2, 3, 4, 5]
    \item [1, 2, 3]
\end{enumerate}

\textbf{Answer: B) [2, 3, 4]}

\textbf{Rationale:} Slicing \texttt{[1:4]} returns elements at indices 1, 2, 3 (end index is exclusive). Values are 2, 3, 4.

\hrule
\vspace{1em}

\subsection*{Question 35}
What is the output?
\begin{lstlisting}[style=codeStyle]
lst = [1, 2, 3]
lst.append([4, 5])
print(len(lst))
\end{lstlisting}

\textbf{Options:}
\begin{enumerate}[label=\Alph*)]
    \item 3
    \item 4
    \item 5
    \item Error
\end{enumerate}

\textbf{Answer: B) 4}

\textbf{Rationale:} \texttt{append()} adds the entire \texttt{[4, 5]} list as a single element. Result: \texttt{[1, 2, 3, [4, 5]]}. Length is 4.

\hrule
\vspace{1em}

\subsection*{Question 36}
What is the output?
\begin{lstlisting}[style=codeStyle]
lst = [1, 2, 3]
lst.extend([4, 5])
print(lst)
\end{lstlisting}

\textbf{Options:}
\begin{enumerate}[label=\Alph*)]
    \item [1, 2, 3, [4, 5]]
    \item [1, 2, 3, 4, 5]
    \item [[1, 2, 3], [4, 5]]
    \item Error
\end{enumerate}

\textbf{Answer: B) [1, 2, 3, 4, 5]}

\textbf{Rationale:} \texttt{extend()} adds each element of the iterable individually to the list, unlike \texttt{append()} which adds the whole object.

\hrule
\vspace{1em}

\subsection*{Question 37}
What is the output?
\begin{lstlisting}[style=codeStyle]
lst = [3, 1, 4, 1, 5, 9, 2, 6]
print(lst.count(1), lst.index(5))
\end{lstlisting}

\textbf{Options:}
\begin{enumerate}[label=\Alph*)]
    \item 2 4
    \item 2 5
    \item 1 4
    \item 1 5
\end{enumerate}

\textbf{Answer: A) 2 4}

\textbf{Rationale:} \texttt{count(1)} returns how many times 1 appears (twice). \texttt{index(5)} returns the first index where 5 is found (index 4).

\hrule
\vspace{1em}

\subsection*{Question 38}
What is the output?
\begin{lstlisting}[style=codeStyle]
lst = [[1, 2], [3, 4], [5, 6]]
print([x for row in lst for x in row if x % 2 == 0])
\end{lstlisting}

\textbf{Options:}
\begin{enumerate}[label=\Alph*)]
    \item [2, 4, 6]
    \item [[2], [4], [6]]
    \item [1, 3, 5]
    \item Error
\end{enumerate}

\textbf{Answer: A) [2, 4, 6]}

\textbf{Rationale:} This nested list comprehension flattens the list and filters even numbers. Iteration order: for each row, for each x in row, keep if x is even.

\hrule
\vspace{1em}

% ============================================
% SECTION 9: TUPLES & SETS (Q39-Q41)
% ============================================
\section{Tuples \& Sets (Questions 39-41)}

\subsection*{Question 39}
What is the output?
\begin{lstlisting}[style=codeStyle]
t = (1, 2, 3)
t[0] = 10
print(t)
\end{lstlisting}

\textbf{Options:}
\begin{enumerate}[label=\Alph*)]
    \item (10, 2, 3)
    \item (1, 2, 3)
    \item TypeError
    \item [10, 2, 3]
\end{enumerate}

\textbf{Answer: C) TypeError}

\textbf{Rationale:} Tuples are immutable. Attempting to modify an element raises \texttt{TypeError: 'tuple' object does not support item assignment}.

\hrule
\vspace{1em}

\subsection*{Question 40}
What is the output?
\begin{lstlisting}[style=codeStyle]
s1 = {1, 2, 3, 4}
s2 = {3, 4, 5, 6}
print(s1 & s2)
\end{lstlisting}

\textbf{Options:}
\begin{enumerate}[label=\Alph*)]
    \item \{1, 2, 3, 4, 5, 6\}
    \item \{3, 4\}
    \item \{1, 2\}
    \item \{5, 6\}
\end{enumerate}

\textbf{Answer: B) \{3, 4\}}

\textbf{Rationale:} The \texttt{\&} operator performs set intersection, returning elements present in both sets. 3 and 4 are common.

\hrule
\vspace{1em}

\subsection*{Question 41}
What is the output?
\begin{lstlisting}[style=codeStyle]
s = {1, 2, 2, 3, 3, 3}
print(len(s))
\end{lstlisting}

\textbf{Options:}
\begin{enumerate}[label=\Alph*)]
    \item 6
    \item 3
    \item 1
    \item Error
\end{enumerate}

\textbf{Answer: B) 3}

\textbf{Rationale:} Sets automatically remove duplicates. The set contains only unique values \{1, 2, 3\}, so length is 3.

\hrule
\vspace{1em}

% ============================================
% SECTION 10: DICTIONARIES (Q42-Q44)
% ============================================
\section{Dictionaries (Questions 42-44)}

\subsection*{Question 42}
What is the output?
\begin{lstlisting}[style=codeStyle]
d = {"a": 1, "b": 2, "c": 3}
print(d.get("d", 0))
\end{lstlisting}

\textbf{Options:}
\begin{enumerate}[label=\Alph*)]
    \item None
    \item 0
    \item KeyError
    \item d
\end{enumerate}

\textbf{Answer: B) 0}

\textbf{Rationale:} \texttt{get()} returns the value for key if it exists, otherwise returns the default value (0 here). Key ``d'' doesn't exist, so 0 is returned.

\hrule
\vspace{1em}

\subsection*{Question 43}
What is the output?
\begin{lstlisting}[style=codeStyle]
d = {"a": 1, "b": 2}
d.update({"b": 3, "c": 4})
print(d)
\end{lstlisting}

\textbf{Options:}
\begin{enumerate}[label=\Alph*)]
    \item \{"a": 1, "b": 2\}
    \item \{"a": 1, "b": 3, "c": 4\}
    \item \{"b": 3, "c": 4\}
    \item Error
\end{enumerate}

\textbf{Answer: B) \{"a": 1, "b": 3, "c": 4\}}

\textbf{Rationale:} \texttt{update()} merges dictionaries. Existing key ``b'' is updated to 3, new key ``c'' is added with value 4.

\hrule
\vspace{1em}

\subsection*{Question 44}
What is the output?
\begin{lstlisting}[style=codeStyle]
d = {"x": 10, "y": 20, "z": 30}
print(list(d.values()))
\end{lstlisting}

\textbf{Options:}
\begin{enumerate}[label=\Alph*)]
    \item ['x', 'y', 'z']
    \item [10, 20, 30]
    \item [('x', 10), ('y', 20), ('z', 30)]
    \item \{10, 20, 30\}
\end{enumerate}

\textbf{Answer: B) [10, 20, 30]}

\textbf{Rationale:} \texttt{values()} returns a view of all values. Converting to list gives \texttt{[10, 20, 30]}. Use \texttt{keys()} for keys, \texttt{items()} for tuples.

\hrule
\vspace{1em}

% ============================================
% SECTION 11: FILE HANDLING & JSON (Q45-Q46)
% ============================================
\section{File Handling \& JSON (Questions 45-46)}

\subsection*{Question 45}
What mode should be used to append to an existing file without overwriting?

\textbf{Options:}
\begin{enumerate}[label=\Alph*)]
    \item "r"
    \item "w"
    \item "a"
    \item "x"
\end{enumerate}

\textbf{Answer: C) "a"}

\textbf{Rationale:} ``a'' is append mode (adds to end). ``r'' is read, ``w'' is write (overwrites), ``x'' is exclusive create (fails if exists).

\hrule
\vspace{1em}

\subsection*{Question 46}
What is the output?
\begin{lstlisting}[style=codeStyle]
import json
data = '{"name": "Alice", "age": 25}'
obj = json.loads(data)
print(type(obj), obj["name"])
\end{lstlisting}

\textbf{Options:}
\begin{enumerate}[label=\Alph*)]
    \item <class 'str'> Alice
    \item <class 'dict'> Alice
    \item <class 'list'> Alice
    \item Error
\end{enumerate}

\textbf{Answer: B) <class 'dict'> Alice}

\textbf{Rationale:} \texttt{json.loads()} parses a JSON string into a Python object. A JSON object becomes a Python dictionary.

\hrule
\vspace{1em}

% ============================================
% SECTION 12: NUMPY (Q47-Q48)
% ============================================
\section{NumPy (Questions 47-48)}

\subsection*{Question 47}
What is the shape of this array?
\begin{lstlisting}[style=codeStyle]
import numpy as np
arr = np.array([[1, 2, 3], [4, 5, 6]])
print(arr.shape)
\end{lstlisting}

\textbf{Options:}
\begin{enumerate}[label=\Alph*)]
    \item (3, 2)
    \item (2, 3)
    \item (6,)
    \item 6
\end{enumerate}

\textbf{Answer: B) (2, 3)}

\textbf{Rationale:} The array has 2 rows and 3 columns. Shape is always (rows, columns) = (2, 3).

\hrule
\vspace{1em}

\subsection*{Question 48}
What is the output?
\begin{lstlisting}[style=codeStyle]
import numpy as np
arr = np.array([1, 2, 3, 4, 5])
print(arr[arr > 2])
\end{lstlisting}

\textbf{Options:}
\begin{enumerate}[label=\Alph*)]
    \item [False, False, True, True, True]
    \item [3, 4, 5]
    \item [1, 2]
    \item Error
\end{enumerate}

\textbf{Answer: B) [3, 4, 5]}

\textbf{Rationale:} \texttt{arr > 2} creates a boolean mask \texttt{[False, False, True, True, True]}. Using this mask as index returns elements where True: \texttt{[3, 4, 5]}.

\hrule
\vspace{1em}

% ============================================
% SECTION 13: PANDAS (Q49-Q50)
% ============================================
\section{Pandas (Questions 49-50)}

\subsection*{Question 49}
What is the output?
\begin{lstlisting}[style=codeStyle]
import pandas as pd
df = pd.DataFrame({'A': [1, 2, 3], 'B': [4, 5, 6]})
print(df['A'].sum())
\end{lstlisting}

\textbf{Options:}
\begin{enumerate}[label=\Alph*)]
    \item [1, 2, 3]
    \item 6
    \item 21
    \item Error
\end{enumerate}

\textbf{Answer: B) 6}

\textbf{Rationale:} \texttt{df['A']} selects column A as a Series \texttt{[1, 2, 3]}. The \texttt{sum()} method returns \texttt{1 + 2 + 3 = 6}.

\hrule
\vspace{1em}

\subsection*{Question 50}
What method is used to get the first 5 rows of a DataFrame?

\textbf{Options:}
\begin{enumerate}[label=\Alph*)]
    \item \texttt{df.first(5)}
    \item \texttt{df.head()}
    \item \texttt{df.top(5)}
    \item \texttt{df[0:5]}
\end{enumerate}

\textbf{Answer: B) df.head()}

\textbf{Rationale:} \texttt{head()} returns the first 5 rows by default (or \texttt{head(n)} for n rows). While \texttt{df[0:5]} also works, \texttt{head()} is the idiomatic Pandas method.

\hrule
\vspace{2em}

% ============================================
% ANSWER KEY
% ============================================
\section*{Quick Answer Key}

\begin{longtable}{|c|c|c|c|c|c|c|c|c|c|}
\hline
\textbf{Q} & \textbf{Ans} & \textbf{Q} & \textbf{Ans} & \textbf{Q} & \textbf{Ans} & \textbf{Q} & \textbf{Ans} & \textbf{Q} & \textbf{Ans} \\
\hline
1 & B & 11 & A & 21 & B & 31 & A & 41 & B \\
\hline
2 & A & 12 & A & 22 & B & 32 & B & 42 & B \\
\hline
3 & B & 13 & B & 23 & B & 33 & B & 43 & B \\
\hline
4 & A & 14 & C & 24 & A & 34 & B & 44 & B \\
\hline
5 & D & 15 & B & 25 & B & 35 & B & 45 & C \\
\hline
6 & A & 16 & B & 26 & B & 36 & B & 46 & B \\
\hline
7 & A & 17 & C & 27 & B & 37 & A & 47 & B \\
\hline
8 & B & 18 & D & 28 & A & 38 & A & 48 & B \\
\hline
9 & A & 19 & B & 29 & B & 39 & C & 49 & B \\
\hline
10 & B & 20 & C & 30 & A & 40 & B & 50 & B \\
\hline
\end{longtable}

\section*{Topic Distribution}
\begin{itemize}
    \item Variables \& Data Types: Q1--Q5 (5 questions)
    \item Operators: Q6--Q10 (5 questions)
    \item Strings: Q11--Q15 (5 questions)
    \item Conditionals: Q16--Q20 (5 questions)
    \item Loops: Q21--Q25 (5 questions)
    \item Functions: Q26--Q30 (5 questions)
    \item Error Handling: Q31--Q33 (3 questions)
    \item Lists: Q34--Q38 (5 questions)
    \item Tuples \& Sets: Q39--Q41 (3 questions)
    \item Dictionaries: Q42--Q44 (3 questions)
    \item File Handling \& JSON: Q45--Q46 (2 questions)
    \item NumPy: Q47--Q48 (2 questions)
    \item Pandas: Q49--Q50 (2 questions)
\end{itemize}

\end{document}

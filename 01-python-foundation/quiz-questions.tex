\documentclass{article}
\usepackage[utf8]{inputenc}
\usepackage{listings}
\usepackage{xcolor}
\usepackage{hyperref}
\usepackage{geometry}
\usepackage{enumitem}
\geometry{margin=0.75in}

% Python code style
\definecolor{codegray}{rgb}{0.95,0.95,0.95}
\definecolor{keyword}{rgb}{0.0,0.0,0.6}
\definecolor{comment}{rgb}{0.0,0.5,0.0}
\definecolor{string}{rgb}{0.64,0.08,0.08}

\lstdefinestyle{codeStyle}{
    language=Python,
    backgroundcolor=\color{codegray},
    commentstyle=\color{comment}\itshape,
    keywordstyle=\color{keyword}\bfseries,
    stringstyle=\color{string},
    basicstyle=\ttfamily\small,
    breaklines=true,
    frame=single,
    showstringspaces=false,
    tabsize=4,
    morekeywords={self, True, False, None, with, as, match, case}
}

\title{Python for Cheminformatics \& Bioinformatics\\Quiz: 100 Questions}
\author{AI-Driven Drug Development Training}
\date{February 2026}

\begin{document}

\maketitle

\section*{Instructions}
\begin{itemize}
    \item Total Questions: 100
    \item Time Allowed: 120 minutes
    \item Each question is worth 1 point (Total: 100 points)
    \item Select the best answer for each question
    \item Write your answers on the answer sheet provided
\end{itemize}

\vspace{1em}
\hrule
\vspace{1em}

% ============================================
% SECTION 1: VARIABLES & DATA TYPES (Q1-Q5)
% ============================================
\section{Variables \& Data Types (Questions 1-5)}

\subsection*{Question 1}
What is the output of the following code?
\begin{lstlisting}[style=codeStyle]
compound_name = "Aspirin"
mw = 180.16
pic50 = "8.28"
print(type(mw) == type(pic50))
\end{lstlisting}

\begin{enumerate}[label=\Alph*)]
    \item True
    \item False
    \item Error
    \item None
\end{enumerate}

\hrule
\vspace{1em}

\subsection*{Question 2}
What will be the value of \texttt{logp}?
\begin{lstlisting}[style=codeStyle]
compound = ("Ibuprofen", 206.28, 3.97)
name, mw, logp = compound
print(logp)
\end{lstlisting}

\begin{enumerate}[label=\Alph*)]
    \item "Ibuprofen"
    \item 206.28
    \item 3.97
    \item Error
\end{enumerate}

\hrule
\vspace{1em}

\subsection*{Question 3}
What is the output?
\begin{lstlisting}[style=codeStyle]
ic50 = None  # Missing measurement
activity = 0  # No inhibition
smiles = ""   # Empty SMILES
print(bool(ic50), bool(activity), bool(smiles))
\end{lstlisting}

\begin{enumerate}[label=\Alph*)]
    \item True True True
    \item False False False
    \item None 0 ""
    \item Error
\end{enumerate}

\hrule
\vspace{1em}

\subsection*{Question 4}
What is the result of \texttt{int(5.8) + int(-2.3)}?

\begin{enumerate}[label=\Alph*)]
    \item 3
    \item 4
    \item 3.5
    \item Error
\end{enumerate}

\hrule
\vspace{1em}

\subsection*{Question 5}
Which statement correctly creates a multi-line DNA sequence string?

\begin{enumerate}[label=\Alph*)]
    \item \texttt{seq = "ATGC\textbackslash nGCTA"}
    \item \texttt{seq = '''ATGC\\GCTA'''}
    \item \texttt{seq = """ATGC\textbackslash nGCTA"""}
    \item All of the above
\end{enumerate}

\hrule
\vspace{1em}

% ============================================
% SECTION 2: OPERATORS (Q6-Q10)
% ============================================
\section{Operators (Questions 6-10)}

\subsection*{Question 6}
What is the output?
\begin{lstlisting}[style=codeStyle]
total_atoms = 17
heavy_atoms = 5
print(total_atoms // heavy_atoms, total_atoms % heavy_atoms, total_atoms / heavy_atoms)
\end{lstlisting}

\begin{enumerate}[label=\Alph*)]
    \item 3 2 3.4
    \item 3.4 2 3
    \item 3 2 3.0
    \item 3.0 2.0 3.4
\end{enumerate}

\hrule
\vspace{1em}

\subsection*{Question 7}
What is the output?
\begin{lstlisting}[style=codeStyle]
mw = 450
logp = 4.5
passes_mw = mw <= 500
passes_logp = logp <= 5
print(passes_mw and passes_logp)
\end{lstlisting}

\begin{enumerate}[label=\Alph*)]
    \item True
    \item False
    \item None
    \item Error
\end{enumerate}

\hrule
\vspace{1em}

\subsection*{Question 8}
What is the value of \texttt{10 ** (9 - 8)}?

\begin{enumerate}[label=\Alph*)]
    \item 1
    \item 10
    \item 100
    \item 0.1
\end{enumerate}

\hrule
\vspace{1em}

\subsection*{Question 9}
What is the output?
\begin{lstlisting}[style=codeStyle]
seq = "ATGCGC"
gc_count = seq.count("G") + seq.count("C")
gc_percent = gc_count / len(seq) * 100
print(gc_percent)
\end{lstlisting}

\begin{enumerate}[label=\Alph*)]
    \item 66.67 (approximately)
    \item 50.0
    \item 33.33
    \item Error
\end{enumerate}

\hrule
\vspace{1em}

\subsection*{Question 10}
What is the output?
\begin{lstlisting}[style=codeStyle]
smiles1 = "CCO"
smiles2 = "CCO"
list1 = [smiles1]
list2 = [smiles2]
print(smiles1 == smiles2, list1 == list2, list1 is list2)
\end{lstlisting}

\begin{enumerate}[label=\Alph*)]
    \item True True True
    \item True True False
    \item True False False
    \item False False False
\end{enumerate}

\hrule
\vspace{1em}

% ============================================
% SECTION 3: STRINGS (Q11-Q15)
% ============================================
\section{Strings (Questions 11-15)}

\subsection*{Question 11}
What is the output?
\begin{lstlisting}[style=codeStyle]
dna = "ATGCGATCG"
print(dna[:3], dna[-3:], dna[::3])
\end{lstlisting}

\begin{enumerate}[label=\Alph*)]
    \item ATG TCG AGC
    \item ATG TCG AGAG
    \item ATG GCG ATG
    \item Error
\end{enumerate}

\hrule
\vspace{1em}

\subsection*{Question 12}
What is the output?
\begin{lstlisting}[style=codeStyle]
dna = "ATGC"
rna = dna.replace("T", "U")
print(rna)
\end{lstlisting}

\begin{enumerate}[label=\Alph*)]
    \item AUGC
    \item ATGC
    \item UAGC
    \item Error
\end{enumerate}

\hrule
\vspace{1em}

\subsection*{Question 13}
What is the output?
\begin{lstlisting}[style=codeStyle]
smiles = "c1ccccc1"  # Benzene
has_ring = any(c.isdigit() for c in smiles)
print(has_ring)
\end{lstlisting}

\begin{enumerate}[label=\Alph*)]
    \item True
    \item False
    \item 1
    \item Error
\end{enumerate}

\hrule
\vspace{1em}

\subsection*{Question 14}
What is the output?
\begin{lstlisting}[style=codeStyle]
name = "Aspirin"
pic50 = 5.28
print(f"{name}: pIC50 = {pic50:.1f}")
\end{lstlisting}

\begin{enumerate}[label=\Alph*)]
    \item Aspirin: pIC50 = 5.28
    \item Aspirin: pIC50 = 5.3
    \item Aspirin: pIC50 = 5
    \item Error
\end{enumerate}

\hrule
\vspace{1em}

\subsection*{Question 15}
What happens when you run this code?
\begin{lstlisting}[style=codeStyle]
seq = "ATGC"
seq[0] = "G"  # Try to create mutation
\end{lstlisting}

\begin{enumerate}[label=\Alph*)]
    \item TypeError - strings are immutable
    \item seq becomes "GTGC"
    \item seq becomes "GATGC"
    \item None
\end{enumerate}

\hrule
\vspace{1em}

% ============================================
% SECTION 4: LISTS (Q16-Q20)
% ============================================
\section{Lists (Questions 16-20)}

\subsection*{Question 16}
What is the output?
\begin{lstlisting}[style=codeStyle]
library_a = ["CCO", "CC", "CCC"]
library_b = library_a  # Reference, not copy!
library_a.append("CCCC")
print(len(library_b))
\end{lstlisting}

\begin{enumerate}[label=\Alph*)]
    \item 3
    \item 4
    \item Error
    \item None
\end{enumerate}

\hrule
\vspace{1em}

\subsection*{Question 17}
What is the output?
\begin{lstlisting}[style=codeStyle]
compounds = ["Aspirin", "Ibuprofen"]
compounds.extend(["Caffeine", "Metformin"])
print(len(compounds))
\end{lstlisting}

\begin{enumerate}[label=\Alph*)]
    \item 2
    \item 3
    \item 4
    \item Error
\end{enumerate}

\hrule
\vspace{1em}

\subsection*{Question 18}
What is the output?
\begin{lstlisting}[style=codeStyle]
pic50_values = [5.2, 6.8, 7.3, 4.9, 8.1]
actives = [p for p in pic50_values if p >= 6.0]
print(len(actives))
\end{lstlisting}

\begin{enumerate}[label=\Alph*)]
    \item 2
    \item 3
    \item 4
    \item 5
\end{enumerate}

\hrule
\vspace{1em}

\subsection*{Question 19}
What is the output?
\begin{lstlisting}[style=codeStyle]
# [MW, LogP, HBD, HBA]
descriptors = [
    [180.16, 1.19, 1, 4],   # Aspirin
    [206.28, 3.97, 1, 2],   # Ibuprofen
    [194.19, -0.07, 0, 6]   # Caffeine
]
print(descriptors[1][1])
\end{lstlisting}

\begin{enumerate}[label=\Alph*)]
    \item 180.16
    \item 1.19
    \item 3.97
    \item 206.28
\end{enumerate}

\hrule
\vspace{1em}

\subsection*{Question 20}
What is the output?
\begin{lstlisting}[style=codeStyle]
compounds = [("Aspirin", 5.2), ("Drug_X", 8.1), ("Ibuprofen", 6.8)]
compounds.sort(key=lambda x: x[1], reverse=True)
print(compounds[0][0])
\end{lstlisting}

\begin{enumerate}[label=\Alph*)]
    \item Aspirin
    \item Drug\_X
    \item Ibuprofen
    \item 8.1
\end{enumerate}

\hrule
\vspace{1em}

% ============================================
% SECTION 5: TUPLES & SETS (Q21-Q25)
% ============================================
\section{Tuples \& Sets (Questions 21-25)}

\subsection*{Question 21}
What happens?
\begin{lstlisting}[style=codeStyle]
compound = ("Aspirin", 180.16, 5.2)
compound[2] = 6.0  # Try to update pIC50
\end{lstlisting}

\begin{enumerate}[label=\Alph*)]
    \item TypeError
    \item compound becomes ("Aspirin", 180.16, 6.0)
    \item None
    \item SyntaxError
\end{enumerate}

\hrule
\vspace{1em}

\subsection*{Question 22}
What is the output?
\begin{lstlisting}[style=codeStyle]
data = (1, 2, 3, 4, 5, 6, 7)
first, *middle, last = data
print(len(middle))
\end{lstlisting}

\begin{enumerate}[label=\Alph*)]
    \item 7
    \item 5
    \item 2
    \item 1
\end{enumerate}

\hrule
\vspace{1em}

\subsection*{Question 23}
What is the output?
\begin{lstlisting}[style=codeStyle]
library_a = {"CMP001", "CMP002", "CMP003"}
library_b = {"CMP002", "CMP003", "CMP004"}
common = library_a & library_b
print(common)
\end{lstlisting}

\begin{enumerate}[label=\Alph*)]
    \item \{``CMP001'', ``CMP004''\}
    \item \{``CMP001'', ``CMP002'', ``CMP003'', ``CMP004''\}
    \item \{``CMP002'', ``CMP003''\}
    \item Error
\end{enumerate}

\hrule
\vspace{1em}

\subsection*{Question 24}
What is the output?
\begin{lstlisting}[style=codeStyle]
scaffolds = {"benzene", "pyridine", "benzene", "furan", "pyridine"}
print(len(scaffolds))
\end{lstlisting}

\begin{enumerate}[label=\Alph*)]
    \item 5
    \item 3
    \item 2
    \item Error
\end{enumerate}

\hrule
\vspace{1em}

\subsection*{Question 25}
What is the output?
\begin{lstlisting}[style=codeStyle]
all_compounds = {"A", "B", "C", "D"}
tested = {"B", "D"}
untested = all_compounds - tested
print(untested)
\end{lstlisting}

\begin{enumerate}[label=\Alph*)]
    \item \{``B'', ``D''\}
    \item \{``A'', ``C''\}
    \item \{``A'', ``B'', ``C'', ``D''\}
    \item Error
\end{enumerate}

\hrule
\vspace{1em}

% ============================================
% SECTION 6: DICTIONARIES (Q26-Q30)
% ============================================
\section{Dictionaries (Questions 26-30)}

\subsection*{Question 26}
What is the output?
\begin{lstlisting}[style=codeStyle]
compound = {"name": "Aspirin", "MW": 180.16}
logp = compound.get("LogP", "N/A")
print(logp)
\end{lstlisting}

\begin{enumerate}[label=\Alph*)]
    \item None
    \item N/A
    \item Error (KeyError)
    \item 0
\end{enumerate}

\hrule
\vspace{1em}

\subsection*{Question 27}
What is printed?
\begin{lstlisting}[style=codeStyle]
props = {"MW": 180.16, "LogP": 1.19, "HBD": 1}
for item in props:
    print(item, end=" ")
\end{lstlisting}

\begin{enumerate}[label=\Alph*)]
    \item 180.16 1.19 1
    \item MW LogP HBD
    \item (``MW'', 180.16) (``LogP'', 1.19) (``HBD'', 1)
    \item Error
\end{enumerate}

\hrule
\vspace{1em}

\subsection*{Question 28}
What is the output?
\begin{lstlisting}[style=codeStyle]
compounds = {
    "Aspirin": {"MW": 180.16, "pIC50": 5.2},
    "Ibuprofen": {"MW": 206.28, "pIC50": 6.8}
}
print(compounds["Aspirin"]["pIC50"])
\end{lstlisting}

\begin{enumerate}[label=\Alph*)]
    \item 180.16
    \item 5.2
    \item 6.8
    \item Error
\end{enumerate}

\hrule
\vspace{1em}

\subsection*{Question 29}
What is the output?
\begin{lstlisting}[style=codeStyle]
import math
ic50_nm = {"A": 10, "B": 100, "C": 1000}
pic50 = {k: 9 - math.log10(v) for k, v in ic50_nm.items()}
print(round(pic50["A"], 1))
\end{lstlisting}

\begin{enumerate}[label=\Alph*)]
    \item 7.0
    \item 8.0
    \item 9.0
    \item 10.0
\end{enumerate}

\hrule
\vspace{1em}

\subsection*{Question 30}
What is the output?
\begin{lstlisting}[style=codeStyle]
codon_table = {"AUG": "M", "UGG": "W", "UAA": "*"}
protein = codon_table.get("AUG", "X") + codon_table.get("UGG", "X")
print(protein)
\end{lstlisting}

\begin{enumerate}[label=\Alph*)]
    \item MW
    \item AUG UGG
    \item XX
    \item Error
\end{enumerate}

\hrule
\vspace{1em}

% ============================================
% SECTION 7: CONTROL FLOW (Q31-Q35)
% ============================================
\section{Control Flow (Questions 31-35)}

\subsection*{Question 31}
What is the output?
\begin{lstlisting}[style=codeStyle]
pic50 = 7.5
if pic50 >= 8:
    print("Highly Active")
elif pic50 >= 6:
    print("Active")
else:
    print("Inactive")
\end{lstlisting}

\begin{enumerate}[label=\Alph*)]
    \item Highly Active
    \item Active
    \item Inactive
    \item Error
\end{enumerate}

\hrule
\vspace{1em}

\subsection*{Question 32}
What is the sum?
\begin{lstlisting}[style=codeStyle]
total = 0
for i in range(0, 10, 2):
    total += i
print(total)
\end{lstlisting}

\begin{enumerate}[label=\Alph*)]
    \item 20
    \item 25
    \item 30
    \item 45
\end{enumerate}

\hrule
\vspace{1em}

\subsection*{Question 33}
What is printed?
\begin{lstlisting}[style=codeStyle]
pic50_values = [5.2, 5.8, 7.5, 6.2, 8.1]
for p in pic50_values:
    if p >= 7.0:
        print(f"Found potent: {p}")
        break
\end{lstlisting}

\begin{enumerate}[label=\Alph*)]
    \item Found potent: 8.1
    \item Found potent: 7.5
    \item Found potent: 7.5\\Found potent: 8.1
    \item Nothing printed
\end{enumerate}

\hrule
\vspace{1em}

\subsection*{Question 34}
What is printed?
\begin{lstlisting}[style=codeStyle]
for compound in ["valid", None, "active", "", "potent"]:
    if not compound:
        continue
    print(compound, end=" ")
\end{lstlisting}

\begin{enumerate}[label=\Alph*)]
    \item valid None active potent
    \item valid active potent
    \item None
    \item Error
\end{enumerate}

\hrule
\vspace{1em}

\subsection*{Question 35}
What is printed?
\begin{lstlisting}[style=codeStyle]
pic50_values = [5.2, 5.8, 5.5]
for p in pic50_values:
    if p >= 6.0:
        print("Found active")
        break
else:
    print("No actives found")
\end{lstlisting}

\begin{enumerate}[label=\Alph*)]
    \item No actives found
    \item Found active
    \item Nothing
    \item Error
\end{enumerate}

\hrule
\vspace{1em}

% ============================================
% SECTION 8: FUNCTIONS (Q36-Q40)
% ============================================
\section{Functions (Questions 36-40)}

\subsection*{Question 36}
What is the output?
\begin{lstlisting}[style=codeStyle]
def classify_activity(pic50, threshold=6.0):
    return "Active" if pic50 >= threshold else "Inactive"

print(classify_activity(5.5))
\end{lstlisting}

\begin{enumerate}[label=\Alph*)]
    \item Active
    \item Inactive
    \item Error
    \item None
\end{enumerate}

\hrule
\vspace{1em}

\subsection*{Question 37}
What is the output?
\begin{lstlisting}[style=codeStyle]
def validate_smiles(smiles):
    if not smiles:
        return
    print("Valid")

result = validate_smiles("")
print(result)
\end{lstlisting}

\begin{enumerate}[label=\Alph*)]
    \item None
    \item Valid\\None
    \item Error
    \item Valid
\end{enumerate}

\hrule
\vspace{1em}

\subsection*{Question 38}
What is the output?
\begin{lstlisting}[style=codeStyle]
def average_pic50(*values):
    return sum(values) / len(values)

print(average_pic50(5.2, 6.8, 7.3))
\end{lstlisting}

\begin{enumerate}[label=\Alph*)]
    \item 19.3
    \item 6.43 (approximately)
    \item Error
    \item (5.2, 6.8, 7.3)
\end{enumerate}

\hrule
\vspace{1em}

\subsection*{Question 39}
What is the output?
\begin{lstlisting}[style=codeStyle]
def create_compound(**props):
    return props

compound = create_compound(name="Aspirin", MW=180.16)
print(type(compound))
\end{lstlisting}

\begin{enumerate}[label=\Alph*)]
    \item \texttt{<class 'tuple'>}
    \item \texttt{<class 'dict'>}
    \item \texttt{<class 'list'>}
    \item Error
\end{enumerate}

\hrule
\vspace{1em}

\subsection*{Question 40}
What is the output?
\begin{lstlisting}[style=codeStyle]
import math
ic50_to_pic50 = lambda ic50_nm: 9 - math.log10(ic50_nm)
print(ic50_to_pic50(100))
\end{lstlisting}

\begin{enumerate}[label=\Alph*)]
    \item 7.0
    \item 8.0
    \item 9.0
    \item 2.0
\end{enumerate}

\hrule
\vspace{1em}

% ============================================
% SECTION 9: FILE & ERROR HANDLING (Q41-Q45)
% ============================================
\section{File \& Error Handling (Questions 41-45)}

\subsection*{Question 41}
What is correct about this code?
\begin{lstlisting}[style=codeStyle]
with open("sequence.fasta", "r") as f:
    content = f.read()
\end{lstlisting}

\begin{enumerate}[label=\Alph*)]
    \item The file is automatically closed after the with block
    \item You must call f.close() manually
    \item The file stays open
    \item Error
\end{enumerate}

\hrule
\vspace{1em}

\subsection*{Question 42}
What happens?
\begin{lstlisting}[style=codeStyle]
with open("compounds.csv", "r") as f:
    f.write("Aspirin,180.16")
\end{lstlisting}

\begin{enumerate}[label=\Alph*)]
    \item Data is written successfully
    \item io.UnsupportedOperation error
    \item FileNotFoundError
    \item Nothing happens
\end{enumerate}

\hrule
\vspace{1em}

\subsection*{Question 43}
What is printed?
\begin{lstlisting}[style=codeStyle]
def parse_smiles(smiles):
    if not smiles:
        raise ValueError("Empty SMILES")
    return smiles

try:
    result = parse_smiles("")
except ValueError:
    print("Error")
print("Done")
\end{lstlisting}

\begin{enumerate}[label=\Alph*)]
    \item Error
    \item Done
    \item Error\\Done
    \item Nothing
\end{enumerate}

\hrule
\vspace{1em}

\subsection*{Question 44}
What is printed?
\begin{lstlisting}[style=codeStyle]
def process_compound():
    try:
        return "Processed"
    finally:
        print("Cleanup")

result = process_compound()
\end{lstlisting}

\begin{enumerate}[label=\Alph*)]
    \item Processed
    \item Cleanup
    \item Both (Cleanup first, then returns ``Processed'')
    \item Error
\end{enumerate}

\hrule
\vspace{1em}

\subsection*{Question 45}
What is printed?
\begin{lstlisting}[style=codeStyle]
try:
    ic50 = float("invalid")
except ValueError:
    print("Invalid IC50")
except TypeError:
    print("Wrong type")
\end{lstlisting}

\begin{enumerate}[label=\Alph*)]
    \item Invalid IC50
    \item Wrong type
    \item Both
    \item Error (uncaught)
\end{enumerate}

\hrule
\vspace{1em}

% ============================================
% SECTION 10: ADVANCED TOPICS (Q46-Q50)
% ============================================
\section{Advanced Topics (Questions 46-50)}

\subsection*{Question 46}
What is the difference?
\begin{lstlisting}[style=codeStyle]
list_comp = [x**2 for x in range(1000000)]
gen_exp = (x**2 for x in range(1000000))
\end{lstlisting}

\begin{enumerate}[label=\Alph*)]
    \item Both use the same memory
    \item Generator uses less memory (lazy evaluation)
    \item List uses less memory
    \item They produce different values
\end{enumerate}

\hrule
\vspace{1em}

\subsection*{Question 47}
What is the output?
\begin{lstlisting}[style=codeStyle]
import math
ic50_values = [10, 100, 1000]
pic50_values = list(map(lambda x: 9 - math.log10(x), ic50_values))
print(pic50_values)
\end{lstlisting}

\begin{enumerate}[label=\Alph*)]
    \item \texttt{[8.0, 7.0, 6.0]}
    \item \texttt{[1.0, 2.0, 3.0]}
    \item \texttt{[10, 100, 1000]}
    \item Error
\end{enumerate}

\hrule
\vspace{1em}

\subsection*{Question 48}
What is the output?
\begin{lstlisting}[style=codeStyle]
pic50_values = [5.2, 6.8, 7.3, 4.9, 8.1]
potent = list(filter(lambda p: p >= 7.0, pic50_values))
print(potent)
\end{lstlisting}

\begin{enumerate}[label=\Alph*)]
    \item \texttt{[5.2, 6.8, 4.9]}
    \item \texttt{[6.8, 7.3, 8.1]}
    \item \texttt{[7.3, 8.1]}
    \item \texttt{[8.1]}
\end{enumerate}

\hrule
\vspace{1em}

\subsection*{Question 49}
What is the output?
\begin{lstlisting}[style=codeStyle]
names = ["Aspirin", "Ibuprofen", "Caffeine"]
pic50s = [5.2, 6.8, 4.8]
compounds = list(zip(names, pic50s))
print(compounds[0])
\end{lstlisting}

\begin{enumerate}[label=\Alph*)]
    \item Aspirin
    \item 5.2
    \item \texttt{("Aspirin", 5.2)}
    \item \texttt{["Aspirin", 5.2]}
\end{enumerate}

\hrule
\vspace{1em}

\subsection*{Question 50}
What is the output?
\begin{lstlisting}[style=codeStyle]
smiles_list = ["CCO", "CC(=O)O", "c1ccccc1"]
for idx, smiles in enumerate(smiles_list, start=1):
    print(f"{idx}: {smiles}")
    break
\end{lstlisting}

\begin{enumerate}[label=\Alph*)]
    \item 0: CCO
    \item 1: CCO
    \item CCO: 1
    \item Error
\end{enumerate}

\hrule
\vspace{1em}

% ============================================
% SECTION 11: CLASSES & OOP (Q51-Q60)
% ============================================
\section{Classes \& Object-Oriented Programming (Questions 51-60)}

\subsection*{Question 51}
What is the output?
\begin{lstlisting}[style=codeStyle]
class Compound:
    def __init__(self, name, mw):
        self.name = name
        self.mw = mw

aspirin = Compound("Aspirin", 180.16)
print(aspirin.name)
\end{lstlisting}

\begin{enumerate}[label=\Alph*)]
    \item Compound
    \item Aspirin
    \item 180.16
    \item Error
\end{enumerate}

\hrule
\vspace{1em}

\subsection*{Question 52}
What is the output?
\begin{lstlisting}[style=codeStyle]
class Molecule:
    atom_count = 0  # Class variable
    
    def __init__(self, atoms):
        Molecule.atom_count += atoms

m1 = Molecule(10)
m2 = Molecule(15)
print(Molecule.atom_count)
\end{lstlisting}

\begin{enumerate}[label=\Alph*)]
    \item 10
    \item 15
    \item 25
    \item Error
\end{enumerate}

\hrule
\vspace{1em}

\subsection*{Question 53}
What is the output?
\begin{lstlisting}[style=codeStyle]
class Drug:
    def __init__(self, name):
        self._name = name  # Protected
    
    @property
    def name(self):
        return self._name.upper()

d = Drug("aspirin")
print(d.name)
\end{lstlisting}

\begin{enumerate}[label=\Alph*)]
    \item aspirin
    \item ASPIRIN
    \item \_name
    \item Error
\end{enumerate}

\hrule
\vspace{1em}

\subsection*{Question 54}
What is the output?
\begin{lstlisting}[style=codeStyle]
class Protein:
    def __len__(self):
        return 150  # amino acids

p = Protein()
print(len(p))
\end{lstlisting}

\begin{enumerate}[label=\Alph*)]
    \item Protein
    \item 150
    \item None
    \item Error
\end{enumerate}

\hrule
\vspace{1em}

\subsection*{Question 55}
What is the output?
\begin{lstlisting}[style=codeStyle]
class Compound:
    def __init__(self, name):
        self.name = name
    
    def __str__(self):
        return f"Compound: {self.name}"

c = Compound("Caffeine")
print(c)
\end{lstlisting}

\begin{enumerate}[label=\Alph*)]
    \item \texttt{<Compound object>}
    \item Compound: Caffeine
    \item Caffeine
    \item Error
\end{enumerate}

\hrule
\vspace{1em}

\subsection*{Question 56}
What is the output?
\begin{lstlisting}[style=codeStyle]
class Molecule:
    pass

class Drug(Molecule):
    pass

d = Drug()
print(isinstance(d, Molecule))
\end{lstlisting}

\begin{enumerate}[label=\Alph*)]
    \item True
    \item False
    \item Drug
    \item Error
\end{enumerate}

\hrule
\vspace{1em}

\subsection*{Question 57}
What is the output?
\begin{lstlisting}[style=codeStyle]
class Compound:
    def describe(self):
        return "Generic compound"

class Drug(Compound):
    def describe(self):
        return "Therapeutic drug"

d = Drug()
print(d.describe())
\end{lstlisting}

\begin{enumerate}[label=\Alph*)]
    \item Generic compound
    \item Therapeutic drug
    \item Both
    \item Error
\end{enumerate}

\hrule
\vspace{1em}

\subsection*{Question 58}
What is the output?
\begin{lstlisting}[style=codeStyle]
class Enzyme:
    @staticmethod
    def calculate_kcat(vmax, enzyme_conc):
        return vmax / enzyme_conc

result = Enzyme.calculate_kcat(100, 0.5)
print(result)
\end{lstlisting}

\begin{enumerate}[label=\Alph*)]
    \item 50.0
    \item 200.0
    \item 0.005
    \item Error
\end{enumerate}

\hrule
\vspace{1em}

\subsection*{Question 59}
What is the output?
\begin{lstlisting}[style=codeStyle]
class Counter:
    count = 0
    
    @classmethod
    def increment(cls):
        cls.count += 1
        return cls.count

print(Counter.increment(), Counter.increment())
\end{lstlisting}

\begin{enumerate}[label=\Alph*)]
    \item 1 1
    \item 1 2
    \item 0 1
    \item Error
\end{enumerate}

\hrule
\vspace{1em}

\subsection*{Question 60}
What is the output?
\begin{lstlisting}[style=codeStyle]
class Compound:
    def __init__(self, name, mw):
        self.name = name
        self.mw = mw
    
    def __eq__(self, other):
        return self.name == other.name

c1 = Compound("Aspirin", 180.16)
c2 = Compound("Aspirin", 180.0)
print(c1 == c2)
\end{lstlisting}

\begin{enumerate}[label=\Alph*)]
    \item True
    \item False
    \item Error
    \item None
\end{enumerate}

\hrule
\vspace{1em}

% ============================================
% SECTION 12: MODULES & IMPORTS (Q61-Q65)
% ============================================
\section{Modules \& Imports (Questions 61-65)}

\subsection*{Question 61}
Which import statement is correct for using only the \texttt{sqrt} function?

\begin{enumerate}[label=\Alph*)]
    \item \texttt{import math.sqrt}
    \item \texttt{from math import sqrt}
    \item \texttt{import sqrt from math}
    \item \texttt{math.import(sqrt)}
\end{enumerate}

\hrule
\vspace{1em}

\subsection*{Question 62}
What is the output?
\begin{lstlisting}[style=codeStyle]
from math import log10 as lg
ic50 = 100  # nM
pic50 = 9 - lg(ic50)
print(pic50)
\end{lstlisting}

\begin{enumerate}[label=\Alph*)]
    \item 7.0
    \item 9.0
    \item 2.0
    \item Error
\end{enumerate}

\hrule
\vspace{1em}

\subsection*{Question 63}
What does \texttt{\_\_name\_\_ == "\_\_main\_\_"} check?

\begin{enumerate}[label=\Alph*)]
    \item If the module is imported
    \item If the script is run directly
    \item If the function is main
    \item If Python version is correct
\end{enumerate}

\hrule
\vspace{1em}

\subsection*{Question 64}
What is the output?
\begin{lstlisting}[style=codeStyle]
import random
random.seed(42)
print(random.randint(1, 10), random.randint(1, 10))
\end{lstlisting}

\begin{enumerate}[label=\Alph*)]
    \item Random numbers each run
    \item Same numbers each run
    \item 42 42
    \item Error
\end{enumerate}

\hrule
\vspace{1em}

\subsection*{Question 65}
What is the output?
\begin{lstlisting}[style=codeStyle]
from collections import Counter
bases = "ATGCGATCGATCG"
counts = Counter(bases)
print(counts["G"])
\end{lstlisting}

\begin{enumerate}[label=\Alph*)]
    \item 3
    \item 4
    \item 5
    \item Error
\end{enumerate}

\hrule
\vspace{1em}

% ============================================
% SECTION 13: LIST METHODS (Q66-Q70)
% ============================================
\section{List Methods (Questions 66-70)}

\subsection*{Question 66}
What is the output?
\begin{lstlisting}[style=codeStyle]
compounds = ["Aspirin", "Ibuprofen", "Caffeine"]
compounds.insert(1, "Metformin")
print(compounds[1])
\end{lstlisting}

\begin{enumerate}[label=\Alph*)]
    \item Aspirin
    \item Metformin
    \item Ibuprofen
    \item Error
\end{enumerate}

\hrule
\vspace{1em}

\subsection*{Question 67}
What is the output?
\begin{lstlisting}[style=codeStyle]
pic50_values = [5.2, 8.1, 6.8, 7.3]
removed = pic50_values.pop(1)
print(removed, len(pic50_values))
\end{lstlisting}

\begin{enumerate}[label=\Alph*)]
    \item 5.2 3
    \item 8.1 3
    \item 8.1 4
    \item Error
\end{enumerate}

\hrule
\vspace{1em}

\subsection*{Question 68}
What is the output?
\begin{lstlisting}[style=codeStyle]
smiles = ["CCO", "CC", "CCC", "CC"]
print(smiles.count("CC"))
\end{lstlisting}

\begin{enumerate}[label=\Alph*)]
    \item 1
    \item 2
    \item 3
    \item 4
\end{enumerate}

\hrule
\vspace{1em}

\subsection*{Question 69}
What is the output?
\begin{lstlisting}[style=codeStyle]
compounds = ["Aspirin", "Caffeine", "Ibuprofen"]
compounds.reverse()
print(compounds[0])
\end{lstlisting}

\begin{enumerate}[label=\Alph*)]
    \item Aspirin
    \item Caffeine
    \item Ibuprofen
    \item Error
\end{enumerate}

\hrule
\vspace{1em}

\subsection*{Question 70}
What is the output?
\begin{lstlisting}[style=codeStyle]
library = ["CCO", "CC(=O)O", "c1ccccc1"]
library_copy = library.copy()
library.append("CCN")
print(len(library_copy))
\end{lstlisting}

\begin{enumerate}[label=\Alph*)]
    \item 3
    \item 4
    \item Error
    \item None
\end{enumerate}

\hrule
\vspace{1em}

% ============================================
% SECTION 14: STRING METHODS (Q71-Q75)
% ============================================
\section{String Methods (Questions 71-75)}

\subsection*{Question 71}
What is the output?
\begin{lstlisting}[style=codeStyle]
smiles = "  CCO  "
print(len(smiles.strip()))
\end{lstlisting}

\begin{enumerate}[label=\Alph*)]
    \item 7
    \item 3
    \item 5
    \item Error
\end{enumerate}

\hrule
\vspace{1em}

\subsection*{Question 72}
What is the output?
\begin{lstlisting}[style=codeStyle]
sequence = "ATGCGATCG"
parts = sequence.split("G")
print(len(parts))
\end{lstlisting}

\begin{enumerate}[label=\Alph*)]
    \item 2
    \item 3
    \item 4
    \item 9
\end{enumerate}

\hrule
\vspace{1em}

\subsection*{Question 73}
What is the output?
\begin{lstlisting}[style=codeStyle]
compounds = ["Aspirin", "Ibuprofen", "Caffeine"]
result = ", ".join(compounds)
print(result)
\end{lstlisting}

\begin{enumerate}[label=\Alph*)]
    \item \texttt{["Aspirin", "Ibuprofen", "Caffeine"]}
    \item Aspirin, Ibuprofen, Caffeine
    \item AspirinIbuprofenCaffeine
    \item Error
\end{enumerate}

\hrule
\vspace{1em}

\subsection*{Question 74}
What is the output?
\begin{lstlisting}[style=codeStyle]
name = "aspirin"
print(name.capitalize(), name.upper())
\end{lstlisting}

\begin{enumerate}[label=\Alph*)]
    \item Aspirin ASPIRIN
    \item ASPIRIN Aspirin
    \item aspirin ASPIRIN
    \item Error
\end{enumerate}

\hrule
\vspace{1em}

\subsection*{Question 75}
What is the output?
\begin{lstlisting}[style=codeStyle]
smiles = "c1ccccc1"
print(smiles.startswith("c"), smiles.endswith("1"))
\end{lstlisting}

\begin{enumerate}[label=\Alph*)]
    \item True True
    \item True False
    \item False True
    \item False False
\end{enumerate}

\hrule
\vspace{1em}

% ============================================
% SECTION 15: DICTIONARY METHODS (Q76-Q80)
% ============================================
\section{Dictionary Methods (Questions 76-80)}

\subsection*{Question 76}
What is the output?
\begin{lstlisting}[style=codeStyle]
compound = {"name": "Aspirin", "MW": 180.16}
compound.update({"LogP": 1.19, "MW": 180.2})
print(compound["MW"])
\end{lstlisting}

\begin{enumerate}[label=\Alph*)]
    \item 180.16
    \item 180.2
    \item Error
    \item None
\end{enumerate}

\hrule
\vspace{1em}

\subsection*{Question 77}
What is the output?
\begin{lstlisting}[style=codeStyle]
props = {"MW": 180.16, "LogP": 1.19}
keys = list(props.keys())
print(keys)
\end{lstlisting}

\begin{enumerate}[label=\Alph*)]
    \item \texttt{["MW", "LogP"]}
    \item \texttt{[180.16, 1.19]}
    \item \texttt{[("MW", 180.16), ("LogP", 1.19)]}
    \item Error
\end{enumerate}

\hrule
\vspace{1em}

\subsection*{Question 78}
What is the output?
\begin{lstlisting}[style=codeStyle]
compound = {"name": "Aspirin", "MW": 180.16}
removed = compound.pop("MW")
print(removed, "MW" in compound)
\end{lstlisting}

\begin{enumerate}[label=\Alph*)]
    \item 180.16 True
    \item 180.16 False
    \item None False
    \item Error
\end{enumerate}

\hrule
\vspace{1em}

\subsection*{Question 79}
What is the output?
\begin{lstlisting}[style=codeStyle]
compound = {"name": "Aspirin"}
compound.setdefault("MW", 180.16)
compound.setdefault("name", "Ibuprofen")
print(compound["name"], compound["MW"])
\end{lstlisting}

\begin{enumerate}[label=\Alph*)]
    \item Ibuprofen 180.16
    \item Aspirin 180.16
    \item Aspirin None
    \item Error
\end{enumerate}

\hrule
\vspace{1em}

\subsection*{Question 80}
What is the output?
\begin{lstlisting}[style=codeStyle]
d1 = {"a": 1, "b": 2}
d2 = {"b": 3, "c": 4}
merged = {**d1, **d2}
print(merged["b"])
\end{lstlisting}

\begin{enumerate}[label=\Alph*)]
    \item 2
    \item 3
    \item 5
    \item Error
\end{enumerate}

\hrule
\vspace{1em}

% ============================================
% SECTION 16: COMPREHENSIONS (Q81-Q85)
% ============================================
\section{Comprehensions (Questions 81-85)}

\subsection*{Question 81}
What is the output?
\begin{lstlisting}[style=codeStyle]
mw_values = [180, 206, 194, 267]
heavy = [mw for mw in mw_values if mw > 200]
print(heavy)
\end{lstlisting}

\begin{enumerate}[label=\Alph*)]
    \item \texttt{[180, 194]}
    \item \texttt{[206, 267]}
    \item \texttt{[206, 194, 267]}
    \item Error
\end{enumerate}

\hrule
\vspace{1em}

\subsection*{Question 82}
What is the output?
\begin{lstlisting}[style=codeStyle]
names = ["aspirin", "ibuprofen"]
upper_names = [n.upper() for n in names]
print(upper_names[0])
\end{lstlisting}

\begin{enumerate}[label=\Alph*)]
    \item aspirin
    \item ASPIRIN
    \item Aspirin
    \item Error
\end{enumerate}

\hrule
\vspace{1em}

\subsection*{Question 83}
What is the output?
\begin{lstlisting}[style=codeStyle]
sequence = "ATGC"
base_set = {base for base in sequence}
print(len(base_set))
\end{lstlisting}

\begin{enumerate}[label=\Alph*)]
    \item 4
    \item 3
    \item 1
    \item Error
\end{enumerate}

\hrule
\vspace{1em}

\subsection*{Question 84}
What is the output?
\begin{lstlisting}[style=codeStyle]
compounds = ["Aspirin", "Ibuprofen", "Caffeine"]
name_len = {c: len(c) for c in compounds}
print(name_len["Caffeine"])
\end{lstlisting}

\begin{enumerate}[label=\Alph*)]
    \item Caffeine
    \item 8
    \item 7
    \item Error
\end{enumerate}

\hrule
\vspace{1em}

\subsection*{Question 85}
What is the output?
\begin{lstlisting}[style=codeStyle]
matrix = [[i*j for j in range(1, 4)] for i in range(1, 3)]
print(matrix[1][2])
\end{lstlisting}

\begin{enumerate}[label=\Alph*)]
    \item 3
    \item 4
    \item 6
    \item Error
\end{enumerate}

\hrule
\vspace{1em}

% ============================================
% SECTION 17: SORTING & ORDERING (Q86-Q90)
% ============================================
\section{Sorting \& Ordering (Questions 86-90)}

\subsection*{Question 86}
What is the output?
\begin{lstlisting}[style=codeStyle]
pic50 = [5.2, 8.1, 6.8, 7.3]
sorted_pic50 = sorted(pic50)
print(sorted_pic50[-1])
\end{lstlisting}

\begin{enumerate}[label=\Alph*)]
    \item 5.2
    \item 8.1
    \item 7.3
    \item Error
\end{enumerate}

\hrule
\vspace{1em}

\subsection*{Question 87}
What is the output?
\begin{lstlisting}[style=codeStyle]
names = ["Caffeine", "Aspirin", "Ibuprofen"]
names.sort()
print(names[0])
\end{lstlisting}

\begin{enumerate}[label=\Alph*)]
    \item Caffeine
    \item Aspirin
    \item Ibuprofen
    \item Error
\end{enumerate}

\hrule
\vspace{1em}

\subsection*{Question 88}
What is the output?
\begin{lstlisting}[style=codeStyle]
compounds = ["CC", "CCCC", "C", "CCC"]
by_length = sorted(compounds, key=len)
print(by_length[0])
\end{lstlisting}

\begin{enumerate}[label=\Alph*)]
    \item CC
    \item CCCC
    \item C
    \item CCC
\end{enumerate}

\hrule
\vspace{1em}

\subsection*{Question 89}
What is the output?
\begin{lstlisting}[style=codeStyle]
data = [(5.2, "A"), (8.1, "B"), (6.8, "C")]
sorted_data = sorted(data, reverse=True)
print(sorted_data[0][1])
\end{lstlisting}

\begin{enumerate}[label=\Alph*)]
    \item A
    \item B
    \item C
    \item 8.1
\end{enumerate}

\hrule
\vspace{1em}

\subsection*{Question 90}
What is the output?
\begin{lstlisting}[style=codeStyle]
values = [3, 1, 4, 1, 5, 9, 2, 6]
print(min(values), max(values))
\end{lstlisting}

\begin{enumerate}[label=\Alph*)]
    \item 1 9
    \item 9 1
    \item 1 6
    \item Error
\end{enumerate}

\hrule
\vspace{1em}

% ============================================
% SECTION 18: TYPE CONVERSION (Q91-Q95)
% ============================================
\section{Type Conversion (Questions 91-95)}

\subsection*{Question 91}
What is the output?
\begin{lstlisting}[style=codeStyle]
mw = "180.16"
print(type(float(mw)))
\end{lstlisting}

\begin{enumerate}[label=\Alph*)]
    \item \texttt{<class 'str'>}
    \item \texttt{<class 'float'>}
    \item \texttt{<class 'int'>}
    \item Error
\end{enumerate}

\hrule
\vspace{1em}

\subsection*{Question 92}
What is the output?
\begin{lstlisting}[style=codeStyle]
values = (1, 2, 3, 2, 1)
unique = set(values)
print(len(unique))
\end{lstlisting}

\begin{enumerate}[label=\Alph*)]
    \item 5
    \item 3
    \item 2
    \item Error
\end{enumerate}

\hrule
\vspace{1em}

\subsection*{Question 93}
What is the output?
\begin{lstlisting}[style=codeStyle]
sequence = "ATGC"
base_list = list(sequence)
print(base_list)
\end{lstlisting}

\begin{enumerate}[label=\Alph*)]
    \item ATGC
    \item \texttt{["ATGC"]}
    \item \texttt{["A", "T", "G", "C"]}
    \item Error
\end{enumerate}

\hrule
\vspace{1em}

\subsection*{Question 94}
What is the output?
\begin{lstlisting}[style=codeStyle]
bases = ["A", "T", "G", "C"]
seq = "".join(bases)
print(seq)
\end{lstlisting}

\begin{enumerate}[label=\Alph*)]
    \item \texttt{["A", "T", "G", "C"]}
    \item A T G C
    \item ATGC
    \item Error
\end{enumerate}

\hrule
\vspace{1em}

\subsection*{Question 95}
What is the output?
\begin{lstlisting}[style=codeStyle]
pic50 = 7.8
print(int(pic50), round(pic50))
\end{lstlisting}

\begin{enumerate}[label=\Alph*)]
    \item 7 7
    \item 7 8
    \item 8 8
    \item Error
\end{enumerate}

\hrule
\vspace{1em}

% ============================================
% SECTION 19: BOOLEAN & NONE (Q96-Q100)
% ============================================
\section{Boolean \& None (Questions 96-100)}

\subsection*{Question 96}
What is the output?
\begin{lstlisting}[style=codeStyle]
result = None
if result is None:
    print("No data")
else:
    print(result)
\end{lstlisting}

\begin{enumerate}[label=\Alph*)]
    \item No data
    \item None
    \item result
    \item Error
\end{enumerate}

\hrule
\vspace{1em}

\subsection*{Question 97}
What is the output?
\begin{lstlisting}[style=codeStyle]
active = True
selective = False
print(active and selective, active or selective)
\end{lstlisting}

\begin{enumerate}[label=\Alph*)]
    \item True True
    \item False True
    \item True False
    \item False False
\end{enumerate}

\hrule
\vspace{1em}

\subsection*{Question 98}
What is the output?
\begin{lstlisting}[style=codeStyle]
values = [0, "", None, "active", 42]
truthy = [v for v in values if v]
print(len(truthy))
\end{lstlisting}

\begin{enumerate}[label=\Alph*)]
    \item 5
    \item 3
    \item 2
    \item 0
\end{enumerate}

\hrule
\vspace{1em}

\subsection*{Question 99}
What is the output?
\begin{lstlisting}[style=codeStyle]
x = 5
y = 10
print(not (x > y), x != y)
\end{lstlisting}

\begin{enumerate}[label=\Alph*)]
    \item True True
    \item False True
    \item True False
    \item False False
\end{enumerate}

\hrule
\vspace{1em}

\subsection*{Question 100}
What is the output?
\begin{lstlisting}[style=codeStyle]
def check_compound(smiles):
    return smiles or "No SMILES provided"

print(check_compound(""))
print(check_compound("CCO"))
\end{lstlisting}

\begin{enumerate}[label=\Alph*)]
    \item "" CCO
    \item No SMILES provided CCO
    \item None CCO
    \item Error
\end{enumerate}

\hrule
\vspace{2em}

\section*{Answer Sheet}
\begin{center}
\begin{tabular}{|c|c|c|c|c|c|c|c|c|c|c|}
\hline
1 & 2 & 3 & 4 & 5 & 6 & 7 & 8 & 9 & 10 \\
\hline
 &  &  &  &  &  &  &  &  &  \\
\hline
\end{tabular}

\vspace{0.5em}

\begin{tabular}{|c|c|c|c|c|c|c|c|c|c|c|}
\hline
11 & 12 & 13 & 14 & 15 & 16 & 17 & 18 & 19 & 20 \\
\hline
 &  &  &  &  &  &  &  &  &  \\
\hline
\end{tabular}

\vspace{0.5em}

\begin{tabular}{|c|c|c|c|c|c|c|c|c|c|c|}
\hline
21 & 22 & 23 & 24 & 25 & 26 & 27 & 28 & 29 & 30 \\
\hline
 &  &  &  &  &  &  &  &  &  \\
\hline
\end{tabular}

\vspace{0.5em}

\begin{tabular}{|c|c|c|c|c|c|c|c|c|c|c|}
\hline
31 & 32 & 33 & 34 & 35 & 36 & 37 & 38 & 39 & 40 \\
\hline
 &  &  &  &  &  &  &  &  &  \\
\hline
\end{tabular}

\vspace{0.5em}

\begin{tabular}{|c|c|c|c|c|c|c|c|c|c|c|}
\hline
41 & 42 & 43 & 44 & 45 & 46 & 47 & 48 & 49 & 50 \\
\hline
 &  &  &  &  &  &  &  &  &  \\
\hline
\end{tabular}

\vspace{0.5em}

\begin{tabular}{|c|c|c|c|c|c|c|c|c|c|c|}
\hline
51 & 52 & 53 & 54 & 55 & 56 & 57 & 58 & 59 & 60 \\
\hline
 &  &  &  &  &  &  &  &  &  \\
\hline
\end{tabular}

\vspace{0.5em}

\begin{tabular}{|c|c|c|c|c|c|c|c|c|c|c|}
\hline
61 & 62 & 63 & 64 & 65 & 66 & 67 & 68 & 69 & 70 \\
\hline
 &  &  &  &  &  &  &  &  &  \\
\hline
\end{tabular}

\vspace{0.5em}

\begin{tabular}{|c|c|c|c|c|c|c|c|c|c|c|}
\hline
71 & 72 & 73 & 74 & 75 & 76 & 77 & 78 & 79 & 80 \\
\hline
 &  &  &  &  &  &  &  &  &  \\
\hline
\end{tabular}

\vspace{0.5em}

\begin{tabular}{|c|c|c|c|c|c|c|c|c|c|c|}
\hline
81 & 82 & 83 & 84 & 85 & 86 & 87 & 88 & 89 & 90 \\
\hline
 &  &  &  &  &  &  &  &  &  \\
\hline
\end{tabular}

\vspace{0.5em}

\begin{tabular}{|c|c|c|c|c|c|c|c|c|c|c|}
\hline
91 & 92 & 93 & 94 & 95 & 96 & 97 & 98 & 99 & 100 \\
\hline
 &  &  &  &  &  &  &  &  &  \\
\hline
\end{tabular}
\end{center}

\vspace{2em}

\begin{center}
\textbf{Name:} \underline{\hspace{6cm}} \hspace{2cm} \textbf{Date:} \underline{\hspace{3cm}}
\end{center}

\end{document}

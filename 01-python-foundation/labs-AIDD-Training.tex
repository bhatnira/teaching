\documentclass{beamer}
\usepackage[utf8]{inputenc}
\usepackage{listings}
\usepackage{xcolor}
\usepackage{hyperref}
\usepackage{tikz}
\usetikzlibrary{shapes.geometric, arrows.meta, positioning}

\usetheme{Madrid}
\hypersetup{breaklinks=true}

% Python code style
\definecolor{codegray}{rgb}{0.95,0.95,0.95}
\definecolor{keyword}{rgb}{0.0,0.0,0.6}
\definecolor{comment}{rgb}{0.0,0.5,0.0}
\definecolor{string}{rgb}{0.64,0.08,0.08}

\lstdefinestyle{codeStyle}{
    language=Python,
    backgroundcolor=\color{codegray},
    commentstyle=\color{comment}\itshape,
    keywordstyle=\color{keyword}\bfseries,
    stringstyle=\color{string},
    basicstyle=\ttfamily\footnotesize,
    breaklines=true,
    frame=single,
    showstringspaces=false,
    tabsize=4,
    morekeywords={self, True, False, None, with, as, match, case}
}

\title{Python for Cheminformatics \& Bioinformatics}
\subtitle{Labs: Hands-On Exercises}
\author{Nirajan Bhattarai}
\institute{AI-Driven Drug Development Training}
\date{February 2026}

\begin{document}

% ============================================
% Title Slide
% ============================================
\begin{frame}
    \titlepage
\end{frame}

% ============================================
% Table of Contents
% ============================================
\begin{frame}{Lab Exercises Overview}
    \tableofcontents
\end{frame}

% ============================================
% SECTION 1: BASICS LABS
% ============================================
\section{Python Basics Labs (Lessons 1--6B)}

% --------------------------------------------
% LAB 1: VARIABLES & DATA TYPES
% --------------------------------------------
\begin{frame}{Lab 1: Variables \& Data Types}
    \small
    \textbf{Objective:} Practice storing and converting molecular and sequence data.
    \vspace{0.5em}
    \begin{block}{Exercise 1.1 -- Compound Data Storage}
        Create variables for a drug compound:
        \begin{itemize}
            \item Name (string): ``Ibuprofen''
            \item SMILES (string): ``CC(C)CC1=CC=C(C=C1)C(C)C(=O)O''
            \item Molecular weight (float): 206.28
            \item pIC50 (float): 6.1
            \item Is active (bool): True if pIC50 $\geq$ 6.0
        \end{itemize}
        Print all values using f-strings.
    \end{block}
\end{frame}

\begin{frame}{Lab 1: Variables \& Data Types (cont.)}
    \small
    \begin{block}{Exercise 1.2 -- DNA Sequence}
        Store a DNA sequence: ``ATGCGATCGATCGATCGATCG''\\
        Calculate and print:
        \begin{itemize}
            \item Sequence length
            \item Number of adenines (A)
            \item Number of thymines (T)
        \end{itemize}
    \end{block}
    \vspace{0.5em}
    \begin{block}{Exercise 1.3 -- Type Conversion}
        Given IC50 = ``5.2'' (string), convert to float and calculate pIC50.\\
        Store the result as both float and formatted string (2 decimals).
    \end{block}
\end{frame}

% --------------------------------------------
% LAB 2: OPERATORS
% --------------------------------------------
\begin{frame}{Lab 2: Operators}
    \small
    \textbf{Objective:} Apply operators for molecular calculations and filtering.
    \vspace{0.5em}
    \begin{block}{Exercise 2.1 -- IC50 Conversion}
        Write code to convert IC50 values from nM to pIC50:\\
        IC50 values: 10.0, 100.0, 1000.0 (nM)\\
        Formula: pIC50 = $-\log_{10}$(IC50 $\times$ $10^{-9}$)\\
        Hint: Use \texttt{import math; math.log10()}
    \end{block}
    \vspace{0.5em}
    \begin{block}{Exercise 2.2 -- Lipinski Check}
        Given: MW=450, LogP=3.5, HBD=2, HBA=8\\
        Check if compound passes Rule of Five:\\
        (MW $\leq$ 500) AND (LogP $\leq$ 5) AND (HBD $\leq$ 5) AND (HBA $\leq$ 10)
    \end{block}
\end{frame}

\begin{frame}{Lab 2: Operators (cont.)}
    \small
    \begin{block}{Exercise 2.3 -- GC Content (Rosalind)}
        Calculate GC content percentage for: ``AGCTATAG''\\
        Formula: GC\% = (G + C) / total $\times$ 100\\
        Expected output: ``GC Content: XX.X\%''
    \end{block}
    \vspace{0.5em}
    \begin{block}{Exercise 2.4 -- Activity Classification}
        Given pIC50 = 7.2, determine if compound is:
        \begin{itemize}
            \item ``Highly potent'' (pIC50 $\geq$ 8)
            \item ``Potent'' (pIC50 $\geq$ 7)
            \item ``Moderate'' (pIC50 $\geq$ 6)
            \item ``Weak'' (pIC50 $<$ 6)
        \end{itemize}
        Use comparison and logical operators.
    \end{block}
\end{frame}

% --------------------------------------------
% LAB 3: STRINGS
% --------------------------------------------
\begin{frame}{Lab 3: Strings}
    \small
    \textbf{Objective:} Manipulate SMILES and biological sequences.
    \vspace{0.5em}
    \begin{block}{Exercise 3.1 -- DNA Transcription}
        Transcribe DNA to RNA:\\
        Input: ``ATGCGATCGATCG''\\
        Replace all T with U\\
        Expected: ``AUGCGAUCGAUCG''
    \end{block}
    \vspace{0.5em}
    \begin{block}{Exercise 3.2 -- Reverse Complement (Rosalind REVC)}
        Generate the reverse complement of DNA:\\
        Input: ``AAAACCCGGT''\\
        Complement: A$\leftrightarrow$T, G$\leftrightarrow$C\\
        Then reverse the string\\
        Expected: ``ACCGGGTTTT''
    \end{block}
\end{frame}

\begin{frame}{Lab 3: Strings (cont.)}
    \small
    \begin{block}{Exercise 3.3 -- SMILES Analysis}
        Analyze SMILES: ``CC(=O)OC1=CC=CC=C1C(=O)O'' (Aspirin)
        \begin{itemize}
            \item Find if it contains a ring (digits indicate ring closure)
            \item Count the number of carbons (C)
            \item Count oxygen atoms (O)
            \item Check if it's aromatic (contains lowercase letters)
        \end{itemize}
    \end{block}
    \vspace{0.5em}
    \begin{block}{Exercise 3.4 -- Nucleotide Count (Rosalind DNA)}
        Count nucleotides in: ``AGCTTTTCATTCTGACTGCAACGGGCAATA''\\
        Print: ``A:X T:X G:X C:X''
    \end{block}
\end{frame}

% --------------------------------------------
% LAB 4: CONDITIONALS
% --------------------------------------------
\begin{frame}{Lab 4: Conditionals}
    \small
    \textbf{Objective:} Implement decision logic for compound classification.
    \vspace{0.5em}
    \begin{block}{Exercise 4.1 -- Drug-Likeness Checker}
        Create a program that checks Lipinski Rule of Five:\\
        Input: MW, LogP, HBD, HBA\\
        Output: Number of violations (0--4)\\
        Print ``Drug-like'' if violations $\leq$ 1, else ``Non-drug-like''
    \end{block}
    \vspace{0.5em}
    \begin{block}{Exercise 4.2 -- Codon Identifier}
        Given a 3-letter codon, identify if it's:
        \begin{itemize}
            \item Start codon: ``ATG''
            \item Stop codon: ``TAA'', ``TAG'', ``TGA''
            \item Other: any other codon
        \end{itemize}
        Use if/elif/else or match-case.
    \end{block}
\end{frame}

\begin{frame}{Lab 4: Conditionals (cont.)}
    \small
    \begin{block}{Exercise 4.3 -- Activity Classifier}
        Classify compound based on pIC50:
        \begin{itemize}
            \item pIC50 $\geq$ 8: ``Highly Active''
            \item 7 $\leq$ pIC50 $<$ 8: ``Active''
            \item 6 $\leq$ pIC50 $<$ 7: ``Moderately Active''
            \item 5 $\leq$ pIC50 $<$ 6: ``Weakly Active''
            \item pIC50 $<$ 5: ``Inactive''
        \end{itemize}
        Test with values: 8.5, 7.2, 6.5, 5.3, 4.1
    \end{block}
\end{frame}

% --------------------------------------------
% LAB 5: LOOPS
% --------------------------------------------
\begin{frame}{Lab 5: Loops}
    \small
    \textbf{Objective:} Process collections of compounds and sequences.
    \vspace{0.5em}
    \begin{block}{Exercise 5.1 -- Batch IC50 Conversion}
        Convert list of IC50 values (nM) to pIC50:\\
        IC50\_list = [1.0, 10.0, 100.0, 1000.0, 10000.0]\\
        Use a for loop to calculate and print each pIC50.
    \end{block}
    \vspace{0.5em}
    \begin{block}{Exercise 5.2 -- Nucleotide Counter (Rosalind DNA)}
        Count all nucleotides in a DNA sequence using a for loop:\\
        seq = ``AGCTTTTCATTCTGACTGCAACGGGCAATATGTCTCTGTGT''\\
        Print counts for A, C, G, T separated by spaces.
    \end{block}
\end{frame}

\begin{frame}{Lab 5: Loops (cont.)}
    \small
    \begin{block}{Exercise 5.3 -- Filter Active Compounds}
        Given pIC50 values: [5.2, 6.8, 7.3, 4.9, 8.1, 5.9, 6.2]\\
        Use a for loop with continue to skip inactive (pIC50 $<$ 6)\\
        Print only active compounds.
    \end{block}
    \vspace{0.5em}
    \begin{block}{Exercise 5.4 -- Find First Potent (While Loop)}
        Given pIC50 values: [5.2, 5.8, 6.1, 7.5, 8.2, 6.8]\\
        Use a while loop with break to find the first ``highly potent'' compound (pIC50 $\geq$ 7.5)\\
        Print its index and value.\\
        \textit{Hint: Use index variable i, increment i += 1}
    \end{block}
\end{frame}

\begin{frame}{Lab 5: Loops (cont.)}
    \small
    \begin{block}{Exercise 5.5 -- Read Until Stop Codon (While Loop)}
        Given codons: [``ATG'', ``CGA'', ``TCG'', ``GGC'', ``TAA'', ``AAA'']\\
        Use a while loop to read codons and build a sequence string.\\
        Stop when you encounter a stop codon (``TAA'', ``TAG'', or ``TGA'').\\
        Print the sequence built before the stop codon.
    \end{block}
    \vspace{0.5em}
    \begin{block}{Exercise 5.6 -- Compound Screening (While Loop)}
        Simulate screening compounds until finding 3 active ones:\\
        Given: [4.5, 5.2, 6.8, 5.1, 7.3, 4.9, 8.1, 5.9]\\
        Use while loop to count actives (pIC50 $\geq$ 6), stop when count reaches 3.\\
        Print how many compounds were screened total.
    \end{block}
\end{frame}

% --------------------------------------------
% LAB 6: FUNCTIONS
% --------------------------------------------
\begin{frame}{Lab 6: Functions}
    \small
    \textbf{Objective:} Create reusable molecular utility functions.
    \vspace{0.5em}
    \begin{block}{Exercise 6.1 -- pIC50 Converter Function}
        Create function: \texttt{ic50\_to\_pic50(ic50\_nm)}\\
        Input: IC50 in nanomolar\\
        Output: pIC50 value\\
        Test with: 10, 100, 1000 nM
    \end{block}
    \vspace{0.5em}
    \begin{block}{Exercise 6.2 -- GC Content Function}
        Create function: \texttt{gc\_content(sequence)}\\
        Input: DNA sequence string\\
        Output: GC percentage (float)\\
        Test with: ``AGCTATAG'', ``GCGCGCGC'', ``ATATAT''
    \end{block}
\end{frame}

\begin{frame}{Lab 6: Functions (cont.)}
    \small
    \begin{block}{Exercise 6.3 -- Lipinski Calculator}
        Create function: \texttt{check\_lipinski(mw, logp, hbd, hba)}\\
        Returns tuple: (passes: bool, violations: int)\\
        Test with multiple compound property sets.
    \end{block}
    \vspace{0.5em}
    \begin{block}{Exercise 6.4 -- Reverse Complement Function}
        Create function: \texttt{reverse\_complement(dna)}\\
        Input: DNA sequence\\
        Output: Reverse complement sequence\\
        Test with Rosalind REVC sample: ``AAAACCCGGT'' $\to$ ``ACCGGGTTTT''
    \end{block}
\end{frame}

% --------------------------------------------
% LAB 6B: ERROR HANDLING
% --------------------------------------------
\begin{frame}{Lab 6B: Error Handling}
    \small
    \textbf{Objective:} Build robust code that handles invalid inputs.
    \vspace{0.5em}
    \begin{block}{Exercise 6B.1 -- Safe IC50 Conversion}
        Modify \texttt{ic50\_to\_pic50()} to handle:
        \begin{itemize}
            \item Negative IC50 values (raise ValueError)
            \item Zero IC50 (raise ValueError)
            \item Non-numeric input (catch TypeError)
        \end{itemize}
        Return None on error and print helpful message.
    \end{block}
    \vspace{0.5em}
    \begin{block}{Exercise 6B.2 -- SMILES Validator}
        Create function that validates SMILES:\\
        Use RDKit: \texttt{Chem.MolFromSmiles(smiles)}\\
        If returns None, raise ValueError with message\\
        Handle with try/except and return valid/invalid status.
    \end{block}
\end{frame}

% ============================================
% SECTION 2: COLLECTIONS LABS
% ============================================
\section{Collections \& Advanced Labs (Lessons 7--12)}

% --------------------------------------------
% LAB 7: LISTS
% --------------------------------------------
\begin{frame}{Lab 7: Lists}
    \small
    \textbf{Objective:} Manage compound libraries using lists.
    \vspace{0.5em}
    \begin{block}{Exercise 7.1 -- Compound Library}
        Create a list of SMILES strings for 5 common drugs.\\
        Perform operations:
        \begin{itemize}
            \item Add a new compound
            \item Remove a compound by value
            \item Insert at specific position
            \item Print first and last compounds
        \end{itemize}
    \end{block}
    \vspace{0.5em}
    \begin{block}{Exercise 7.2 -- pIC50 Statistics}
        Given: [5.2, 6.8, 7.3, 4.9, 8.1, 5.9, 6.2, 7.8]\\
        Calculate: min, max, sorted list, count of actives ($\geq$6)
    \end{block}
\end{frame}

% --------------------------------------------
% LAB 7B: TUPLES & SETS
% --------------------------------------------
\begin{frame}{Lab 7B: Tuples \& Sets}
    \small
    \begin{block}{Exercise 7B.1 -- Compound Records}
        Create tuples for 3 compounds: (name, SMILES, pIC50)\\
        Unpack each tuple and print formatted output.\\
        Try to modify a tuple value -- observe the error.
    \end{block}
    \vspace{0.5em}
    \begin{block}{Exercise 7B.2 -- Library Comparison}
        Library A: \{``CMP001'', ``CMP002'', ``CMP003'', ``CMP004''\}\\
        Library B: \{``CMP003'', ``CMP004'', ``CMP005'', ``CMP006''\}\\
        Find:
        \begin{itemize}
            \item All unique compounds (union)
            \item Common compounds (intersection)
            \item Compounds only in A
            \item Compounds only in B
        \end{itemize}
    \end{block}
\end{frame}

% --------------------------------------------
% LAB 8: ADVANCED LISTS
% --------------------------------------------
\begin{frame}{Lab 8: List Comprehensions}
    \small
    \textbf{Objective:} Use comprehensions for elegant data transformations.
    \vspace{0.5em}
    \begin{block}{Exercise 8.1 -- Filter Active Compounds}
        Given pIC50 = [5.2, 6.8, 7.3, 4.9, 8.1, 5.9]\\
        Use list comprehension to filter pIC50 $\geq$ 6.0
    \end{block}
    \vspace{0.5em}
    \begin{block}{Exercise 8.2 -- Batch Conversion}
        Convert IC50 list [10, 100, 1000] to pIC50 using:\\
        a) List comprehension\\
        b) map() with lambda
    \end{block}
    \vspace{0.5em}
    \begin{block}{Exercise 8.3 -- Conditional Comprehension}
        Create list of tuples: [(pIC50, ``Active'') if pIC50 $\geq$ 6 else (pIC50, ``Inactive'')]\\
        for values [5.2, 6.8, 7.3, 4.9, 8.1]
    \end{block}
\end{frame}

% --------------------------------------------
% LAB 9: DICTIONARIES
% --------------------------------------------
\begin{frame}{Lab 9: Dictionaries}
    \small
    \textbf{Objective:} Build compound databases using dictionaries.
    \vspace{0.5em}
    \begin{block}{Exercise 9.1 -- Compound Database}
        Create a dict of compounds with nested properties:\\
        Key: compound name\\
        Value: dict with SMILES, MW, pIC50, is\_active
    \end{block}
    \vspace{0.5em}
    \begin{block}{Exercise 9.2 -- Codon Table (Rosalind)}
        Create a dict mapping codons to amino acids:\\
        ``ATG'' $\to$ ``M'', ``TGG'' $\to$ ``W'', ``TAA'' $\to$ ``Stop'', etc.\\
        Use to translate a short sequence.
    \end{block}
    \vspace{0.5em}
    \begin{block}{Exercise 9.3 -- Dict Comprehension}
        Filter the compound database to only active compounds\\
        using dict comprehension: \{k:v for k,v in ... if ...\}
    \end{block}
\end{frame}

% --------------------------------------------
% LAB 10: FILE HANDLING
% --------------------------------------------
\begin{frame}{Lab 10: File Handling}
    \small
    \textbf{Objective:} Read and write compound and sequence files.
    \vspace{0.5em}
    \begin{block}{Exercise 10.1 -- Write Compound CSV}
        Create a CSV file with columns: name, SMILES, pIC50\\
        Write data for 5 compounds using \texttt{with open()}.
    \end{block}
    \vspace{0.5em}
    \begin{block}{Exercise 10.2 -- Read FASTA}
        Create a simple FASTA parser:\\
        Read file, extract header (lines starting with $>$)\\
        Concatenate sequence lines\\
        Return dict: \{header: sequence\}
    \end{block}
    \vspace{0.5em}
    \begin{block}{Exercise 10.3 -- Filter and Export}
        Read compound CSV, filter active compounds (pIC50 $\geq$ 6)\\
        Write filtered results to new file ``actives.csv''
    \end{block}
\end{frame}

% --------------------------------------------
% LAB 11: NUMPY
% --------------------------------------------
\begin{frame}{Lab 11: NumPy}
    \small
    \textbf{Objective:} Use NumPy for numerical molecular data.
    \vspace{0.5em}
    \begin{block}{Exercise 11.1 -- Descriptor Matrix}
        Create a 2D array of molecular descriptors:\\
        Rows = compounds, Columns = [MW, LogP, HBD, HBA]\\
        Calculate mean and std for each descriptor (column).
    \end{block}
    \vspace{0.5em}
    \begin{block}{Exercise 11.2 -- Normalization}
        Normalize descriptor values to 0-1 range:\\
        Formula: (x - min) / (max - min)\\
        Use vectorized operations (no loops).
    \end{block}
    \vspace{0.5em}
    \begin{block}{Exercise 11.3 -- Boolean Filtering}
        Filter compounds where MW $<$ 500 AND LogP $<$ 5\\
        Use boolean indexing.
    \end{block}
\end{frame}

% --------------------------------------------
% LAB 11B: PANDAS
% --------------------------------------------
\begin{frame}{Lab 11B: Pandas}
    \small
    \textbf{Objective:} Analyze compound datasets with Pandas.
    \vspace{0.5em}
    \begin{block}{Exercise 11B.1 -- Create Compound DataFrame}
        Create DataFrame with: Name, SMILES, MW, LogP, pIC50\\
        Add 5+ compounds. Add column for activity class.
    \end{block}
    \vspace{0.5em}
    \begin{block}{Exercise 11B.2 -- Data Analysis}
        Calculate: mean pIC50, count by activity class\\
        Filter drug-like compounds (MW $<$ 500, LogP $<$ 5)\\
        Sort by pIC50 descending.
    \end{block}
    \vspace{0.5em}
    \begin{block}{Exercise 11B.3 -- GroupBy Analysis}
        Group compounds by activity class\\
        Calculate mean MW and LogP per group\\
        Export results to CSV.
    \end{block}
\end{frame}

% --------------------------------------------
% LAB 12: JSON & REGEX
% --------------------------------------------
\begin{frame}{Lab 12: JSON \& Regex}
    \small
    \begin{block}{Exercise 12.1 -- Parse PubChem JSON}
        Parse JSON compound data:\\
        \texttt{\{"CID": 2244, "name": "Aspirin", "MW": 180.16\}}\\
        Extract and print each field.
    \end{block}
    \vspace{0.5em}
    \begin{block}{Exercise 12.2 -- Find Restriction Sites}
        Use regex to find all occurrences of ``GAATTC'' (EcoRI site)\\
        in sequence: ``ATGAATTCGCGAATTCTA''\\
        Print positions of each match.
    \end{block}
    \vspace{0.5em}
    \begin{block}{Exercise 12.3 -- SMILES Validation}
        Use regex to check if SMILES contains:
        \begin{itemize}
            \item Aromatic ring (lowercase c, n, o, s)
            \item Ring closure (digits)
            \item Double bond (=)
        \end{itemize}
    \end{block}
\end{frame}

% ============================================
% SECTION 3: ROSALIND CHALLENGES
% ============================================
\section{Rosalind Bioinformatics Challenges}

\begin{frame}{Rosalind Challenges}
    \small
    \textbf{Complete these Rosalind.info problems:}
    \vspace{0.5em}
    \begin{enumerate}
        \item \textbf{DNA} -- Counting DNA Nucleotides
        \item \textbf{RNA} -- Transcribing DNA into RNA
        \item \textbf{REVC} -- Complementing a Strand of DNA
        \item \textbf{GC} -- Computing GC Content
        \item \textbf{HAMM} -- Counting Point Mutations
        \item \textbf{PROT} -- Translating RNA into Protein
        \item \textbf{SUBS} -- Finding a Motif in DNA
        \item \textbf{CONS} -- Consensus and Profile
    \end{enumerate}
    \vspace{0.5em}
    \textbf{Submission:} Upload your solutions to GitHub with:\\
    - Clear function documentation\\
    - Test cases with sample data\\
    - README explaining approach
\end{frame}

% ============================================
% SECTION 4: CHEMINFORMATICS PROJECT
% ============================================
\section{Cheminformatics Mini-Project}

\begin{frame}{Mini-Project: QSAR Data Prep Pipeline}
    \small
    \textbf{Objective:} Build a complete data preparation pipeline.
    \vspace{0.5em}
    \textbf{Tasks:}
    \begin{enumerate}
        \item Read compound CSV with SMILES and IC50 values
        \item Validate all SMILES using RDKit
        \item Convert IC50 (nM) to pIC50
        \item Calculate Lipinski descriptors (MW, LogP, HBD, HBA)
        \item Add activity class column (Active/Inactive)
        \item Filter drug-like compounds
        \item Export clean dataset to CSV
        \item Generate summary statistics
    \end{enumerate}
    \vspace{0.3em}
    \textbf{Deliverables:}
    \begin{itemize}
        \item Python script with documented functions
        \item Output CSV file
        \item Summary report (mean/std of descriptors)
    \end{itemize}
\end{frame}

\begin{frame}{Mini-Project: Hints}
    \small
    \textbf{Useful RDKit Functions:}
    \begin{itemize}
        \item \texttt{Chem.MolFromSmiles(smiles)} -- parse SMILES
        \item \texttt{Descriptors.MolWt(mol)} -- molecular weight
        \item \texttt{Descriptors.MolLogP(mol)} -- LogP
        \item \texttt{Descriptors.NumHDonors(mol)} -- H-bond donors
        \item \texttt{Descriptors.NumHAcceptors(mol)} -- H-bond acceptors
    \end{itemize}
    \vspace{0.5em}
    \textbf{Pandas Operations:}
    \begin{itemize}
        \item \texttt{pd.read\_csv()} -- read input
        \item \texttt{df.apply()} -- apply function to column
        \item \texttt{df.describe()} -- summary statistics
        \item \texttt{df.to\_csv()} -- export results
    \end{itemize}
\end{frame}

% ============================================
% CLOSING
% ============================================
\section{Summary}

\begin{frame}{Lab Summary}
    \small
    \textbf{Basics Labs (1--6B):}
    \begin{itemize}
        \item Variables, operators, strings for molecular data
        \item Conditionals for classification
        \item Loops for batch processing
        \item Functions for reusable utilities
        \item Error handling for robust code
    \end{itemize}
    \vspace{0.5em}
    \textbf{Collections Labs (7--12):}
    \begin{itemize}
        \item Lists, tuples, sets, dicts for compound storage
        \item Comprehensions for elegant transformations
        \item File I/O for CSV and FASTA
        \item NumPy/Pandas for data analysis
        \item JSON/Regex for APIs and pattern matching
    \end{itemize}
    \vspace{0.5em}
    \textbf{Projects:}
    \begin{itemize}
        \item Rosalind bioinformatics challenges
        \item QSAR data prep pipeline
    \end{itemize}
\end{frame}

\begin{frame}{Resources}
    \small
    \textbf{Practice Platforms:}
    \begin{itemize}
        \item Rosalind.info -- Bioinformatics problems
        \item LeetCode -- General programming
        \item Kaggle -- Data science competitions
    \end{itemize}
    \vspace{0.5em}
    \textbf{Cheminformatics Data:}
    \begin{itemize}
        \item ChEMBL -- Bioactivity database
        \item PubChem -- Chemical compounds
        \item ZINC -- Virtual screening library
    \end{itemize}
    \vspace{0.5em}
    \textbf{Documentation:}
    \begin{itemize}
        \item RDKit: rdkit.org
        \item Pandas: pandas.pydata.org
        \item NumPy: numpy.org
    \end{itemize}
    \vspace{1em}
    \centering
    \textbf{\Large Good luck with your exercises!}
\end{frame}

\end{document}
